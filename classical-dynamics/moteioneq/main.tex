\documentclass{scrartcl}
\usepackage[left=2cm, right=2cm, top=2cm, bottom=2cm]{geometry}
\usepackage{amsmath}
\usepackage{amssymb}
\usepackage{amsthm}
\usepackage{blindtext}
\usepackage{datetime}
\usepackage{fontspec}
\usepackage{graphicx}
\usepackage{hyperref}
\usepackage{mathrsfs}
\usepackage{mathtools}
\usepackage{pgf,tikz,pgfplots}

\usepackage[headsepline]{scrlayer-scrpage}
\newcommand{\doctitle}{입자의 운동 방정식 문제풀이}
\clearpairofpagestyles
\ohead{\thepage}
\ihead{\doctitle}

\hypersetup{
    colorlinks=true,
    linkcolor=black,
    urlcolor=blue,
}
\pgfplotsset{compat=1.15}
\usetikzlibrary{arrows}
\newtheorem{theorem}{Theorem}

\setmainfont{Noto Serif CJK KR}
\setsansfont{NanumSquare}
\newfontfamily\sectionfont{NanumSquare Bold}
\addtokomafont{section}{\sectionfont\mdseries}
\addtokomafont{subsection}{\sectionfont\mdseries}
\addtokomafont{subsubsection}{\sectionfont\mdseries}
\addtokomafont{paragraph}{\sectionfont\mdseries}
\addtokomafont{title}{\sectionfont\mdseries}
\addtokomafont{author}{\sectionfont}
\title{\doctitle}
\author{Project Eclipse (손량)}
\date{Last compiled on: \today, \currenttime}

\newcommand{\Lim}[1]{\lim_{#1\to\infty}}
\newcommand{\para}{\mathbin{\!/\mkern-5mu/\!}}
\newcommand{\Seg}[1]{\overline{#1}}
\newcommand{\Line}[1]{\overleftrightarrow{#1}}
\newcommand{\Ray}[1]{\overrightarrow{#1}}
\newcommand{\infsum}[1]{\sum^\infty_{#1}}
\newcommand{\mat}[3]{\begin{pmatrix}
#1_{11} & #1_{12} & \cdots & #1_{1#3} \\
#1_{21} & #1_{22} & \cdots & #1_{2#3} \\
\vdots & \vdots & \ddots & \vdots \\
#1_{#2 1} & #1_{#2 2} & \cdots & #1_{#2#3} \\
\end{pmatrix}}
\newcommand{\un}[1]{\ensuremath{\ \mathrm{#1}}}

\begin{document}
\maketitle

\section{예제 2.4}
운동 방정식을 세우면 다음과 같다.
\[-kmv=m\frac{dv}{dt}\]
변수분리법으로 적분해 보자.
\[\int^t_0-k\,dt=\int^v_{v_0}\frac{dv}{v}\longrightarrow v(t)=v_0 e^{-kt}\]
저항력에 의해 속도가 지수적으로 감소함을 알 수 있다. 속도를 적분하면 변위도 구할 수 있다.
\[x(t)=\int^t_0 v\,dt=-\frac{1}{k}v_0 e^{-kt}+\frac{v_0}{k}+x_0\]

\section{문제 2}
우선 운동 방정식부터 세우자.
\[F=mg-bv=m\frac{dv}{dt}\]
속도를 구하기 위해, 변수분리법으로 적분하자.
\[\int^t_0\frac{dt}{m}=\int^v_0\frac{dv}{mg-bv}\longrightarrow\frac{t}{m}=\left[-\frac{1}{b}\ln|mg-bv|\right]^v_0=-\frac{1}{b}\ln\left(1-\frac{bv}{mg}\right)\]
따라서 속도는 다음과 같이 구해진다.
\[v=\frac{mg}{b}\left(1-e^{-\frac{bt}{m}}\right)\]

\section{문제 3}
처음부터 부피가 들어갈 곳에 \(4\pi r^3/3\)을 대입하고 풀면 정신건강에 해롭다. 우선 부피를 \(V\)라 두고 풀자. 운동 방정식을 세우면
\[F=\rho Vg-\rho_f Vg-6\pi\eta rv=m\frac{dv}{dt}\]
변수 분리법으로 적분해 보자.
\[\int^v_0 \frac{dv}{\rho Vg-\rho_f Vg-6\pi\eta rv}=\int^t_0\frac{dt}{m}\longrightarrow\left[-\frac{1}{6\pi\eta r}\ln\left|(\rho-\rho_f)Vg-6\pi\eta rv\right|\right]^v_0=\frac{t}{m}\]
식을 정리하면 다음과 같은 결과를 얻는다.\footnote{\(e^x\) 형태로 썼더니 너무 작아서 \(\exp(x)\) 형태로 표현하였다. 예상했겠지만 \(e^x=\exp(x)\)이다.}
\[v=\frac{(\rho-\rho_f)Vg}{6\pi\eta r}\left[1-\exp\left(-\frac{6\pi\eta rt}{m}\right)\right]\]
종단 속도는 구한 속도에 \(t\to\infty\)인 극한을 취하면 얻을 수 있다.
\[v_t=\lim_{t\to\infty} v=\frac{(\rho-\rho_f)Vg}{6\pi\eta r}=\frac{(\rho-\rho_f)g}{6\pi\eta r}\cdot\frac{4}{3}\pi r^3=\frac{2(\rho-\rho_f)r^2g}{9\eta}\]
따라서 구하는 반지름은 다음과 같다.
\[r=\sqrt{\frac{9\eta v_t}{2(\rho-\rho_f)g}}\]

\section{예제 2.5}
우선 운동 방정식을 세우자.\footnote{이 문제는 특히 방향에 유의해야 하는데, 대부분의 어려움은 사실 +방향을 아래쪽으로 잡아서 피할 수 있다. 우리의 머리는 `음의 속도'를 생각할 때 자주 헷갈리기 때문에, 물체가 운동하는 방향을 +로 잡는 것이 여러 모로 정신건강에 편하고 실수도 줄일 수 있다.}
\[F=-mg-kmv=m\frac{dv}{dt}\]
\(kmv\)에 마이너스를 붙인 이유는 \(v<0\)이기 때문에 힘의 방향을 윗방향으로 맞추기 위해서이다. 적분하자.
\[\int^t_0\frac{dt}{m}=\int^v_{v_0}\frac{dv}{-mg-kmv}\longrightarrow\frac{t}{m}=-\frac{1}{km}\left[\ln|-mg-kmv|\right]^v_{v_0}\]
식을 정리해 속도를 구하면 다음과 같다.
\[v=-\frac{g}{k}+\frac{g+kv_0}{k}e^{-kt}\]
한번 더 적분하면 변위를 구할 수 있다.
\begin{align*}
z(t)&=h+\int^t_0\left(-\frac{g}{k}+\frac{g+kv_0}{k}e^{-kt}\right)dt=h+\left[-\frac{g}{k}t-\frac{g+kv_0}{k^2}e^{-kt}\right]^t_0 \\
&=h-\frac{g}{k}t-\frac{g+kv_0}{k^2}e^{-kt}+\frac{g+kv_0}{k^2}
\end{align*}

\section{문제 4}
운동 방정식을 세우자.
\[F=-kv-f=m\frac{dv}{dt}\]
변수분리법으로 적분하자.
\[\int^t_0\frac{dt}{m}=\int^v_{v_0}\frac{dv}{-kv-f}\longrightarrow\frac{t}{m}=-\frac{1}{k}\left[\ln|-kv-f|\right]^v_{v_0}=-\frac{1}{k}\ln\frac{kv+f}{kv_0+f}\]
식을 정리하면... 이라고 하면서 \(v\)에 대해 정리하기 전에, 문제에서 묻는 것을 살펴보자. 그냥 여기 나온 식에서 \(t_\text{정지}\)를 대입했을 때 \(v=0\)임을 이용하여 풀면 된다.
\[\frac{t_\text{정지}}{m}=-\frac{1}{k}\ln\frac{f}{kv_0+f}\]
따라서 구하는 값은 다음과 같다.
\[t_\text{정지}=-\frac{m}{k}\ln\frac{f}{kv_0+f}\]

\section{문제 5}
이 문제에서는 질점이 아니라 강체를 보고 해결해야 하므로, 마찰력이 작용하는 구간에 얼마나 `걸쳐' 있는지 등을 따져보아야 한다. 이를 처리하는 방법은 여러 가지가 있지만 여기서는 가장 간단하게, 강체의 선밀도 \(\lambda\)를 도입할 것이다.\footnote{물론 이 문제는 3차원을 가정하지만, 강체의 밀도가 균일하기 때문에 운동 방향과 평행한 모서리를 기준으로 두고 선밀도를 정의해도 큰 문제가 없다.} 물체가 마찰력이 작용하지 않는 구간에서 얼마나 운동했는지는 최종 결과에 영향을 미치지 않기 때문에, 물체의 오른쪽이 경계에 있는 시점을 \(t=0\)으로 두고, 마찰력이 있는 구간에 걸친 물체의 길이를 \(x\)라 두자. 이때 물체에 작용하는 마찰력의 크기는 다음과 같다.
\[f=\mu_k\lambda xg\]
여기서부터 문제를 푸는 방법이 두 가지로 구분되는데, 두 가지 방법 모두를 사용하여 해결해 보자.

\subsection{마찰력이 한 일 구하기}
알짜일이 한 일은 운동에너지 변화량과 같으므로, 마찰력이 한 일은 물체가 초기에 갖고 있었던 운동에너지 크기와 같을 것이다. 이때 다음과 같은 식을 세울 수 있다.
\[\frac{1}{2}mv^2=\int^L_0\mu_k\lambda gx\,dx=\frac{1}{2}\mu_k\lambda gL^2\]
\(m=L\lambda\)이므로 운동마찰계수는 다음과 같이 구할 수 있다.
\[\frac{1}{2}mv^2=\frac{1}{2}\mu_k mgL\longrightarrow\mu_k=\frac{v^2}{gL}\]
\[\]

\subsection{운동 방정식 풀기}
앞선 방법보다 더 복잡하지만, 운동 방정식을 세워서 풀 수도 있다. 운동방정식은 다음과 같다.
\[-\mu_k\lambda xg=m\frac{dv}{dt}=m\ddot{x}\longrightarrow \ddot{x}+\frac{\mu_k g}{L}x=0\]
어디서 많이 본 방정식이다. 방정식의 해는 \(x=A\sin\omega t\) 꼴로 나타낼 수 있고\footnote{보통 초기위상까지 필요하지만, 여기에서는 \(t=0\)에서 \(x=0\)임이 명백하기 때문에 초기 위상 없이 시작하는 것이 깔끔하다.}, 문제의 조건에 따라, \(\dot{x}=0\)일때 \(x=L\)이므로 \(A=L\)이라는 사실을 알 수 있으므로 방정식의 해는 다음과 같다.
\[x=L\sin\sqrt{\frac{\mu_k g}{L}}t\]
미분하면 다음과 같다.
\[\dot{x}=L\sqrt{\frac{\mu_k g}{L}}\cos\sqrt{\frac{\mu_k g}{L}}t\]
또한, \(t=0\)에서 \(\dot{x}=v\)이므로 운동마찰계수를 구할 수 있다.
\[v=L\sqrt{\frac{\mu_k g}{L}}\longrightarrow\mu_k=\frac{v^2}{gL}\]

\section{문제 6}
우선 운동방정식부터 세우자.
\[F=-bv^2=m\frac{dv}{dt}\]
늘 하던 대로, 변수분리법으로 적분하명 다음과 같다.
\[\int^t_0\frac{b}{m}dt=\int^v_{v_0}-\frac{dv}{v^2}\longrightarrow\frac{b}{m}t=\frac{1}{v}-\frac{1}{v_0}\]
속도를 구하면 다음과 같다.
\[v=\frac{mv_0}{bv_0t+m}\]
한번 더 적분해주면 변위를 구할 수 있다.
\[x=\int^t_0 v\,dt=\left[\frac{m}{b}\ln|bv_0 t+m|\right]^t_0=\frac{m}{b}\ln\left(\frac{bv_0 t}{m}+1\right)\]

\section{쓸데 없는 이야기: 차원 분석}
문제들을 풀면서 느꼈겠지만 미분방정식을 풀고 정리하는 과정에서 계산이 많기 때문에 실수하기 쉽다. 실수를 줄이기 위해서는 연습하는게 답이긴 하지만, 차원 분석이라는 방법을 통해 구한 답이 논리적으로 맞는지 간단하게 확인할 수는 있다.

차원 분석은 쉽게 말해, 단위를 확인하는 방법이라고 할 수 있다. 속도를 구하는 문제에서 답을 구했는데, 단위가 \un{m/s^2}으로 나오면 중간에 잘못되었다는 것을 알 수 있을 것이다. 그런데 위에서 문제풀이를 한 것과 같이, 보통 계산을 할 때에는 문자로 계산하고 딱히 단위를 붙이지 않기 때문에 단위 확인 대신, 차원 분석을 하는 것이다.

물리량의 `기본 단위', 또는 `차원'은 질량(\(M\)), 길이(\(L\)), 시간(\(T\))이고, 속도나 가속도, 힘 등 단위는 이들 차원을 조합해 만든다. 예를 들어서 속도는 단위 시간당 거리 (길이)이므로 다음과 같이 쓸 수 있다. 물리량에 괄호를 붙여서 등호로 엮는 표기법은 일종의 `단위 방정식'이라고 볼 수 있다.
\[[v]=[LT^{-1}]\]
힘은 질량과 가속도의 곱\footnote{질량 변화량과 속도의 곱으로 보아도 똑같은 결과를 얻는다.}, 가속도는 단위 시간당 시간이므로 다음이 성립한다.
\[[F]=[ma]=[MLT^{-2}]\]
이제 차원 분석을 사용하여 문제 6에서 구한 속도를\footnote{예리한 사람들은 변위에 로그가 들어가서 여기서 하는 방법대로 안 된다는 것을 눈치챘을 탠데, 보통 로그나 지수는 `차원이 없는' 물리량으로 생각하고 풀어도 무방하다.} 검증해보자. 우선 \(b\)의 차원을 알아야 하는데, \(F=bv^2\)이므로
\[[F]=[bv^2]=[b][L^2T^{-2}]=[MLT^{-2}]\longrightarrow[b]=[ML^{-1}]\]
따라서 앞서 구한 속도의 차원을 보면
\[[v]=\left[\frac{mv_0}{bv_0t+m}\right]=\frac{[MLT^{-1}]}{[ML^{-1}][LT^{-1}][T]+[M]}=[LT^{-1}]\]
구한 값이 속도의 차원을 가짐을 알 수 있고, 따라서 이 답은 적어도 단위는 맞는다고 할 수 있다. 만약 \(v\)를 잘못 구했다면 어떨까? 예를 들어, 중간에 계산 실수를 해서 \(v\) 대신 \(v'\)을 다음과 같이 구했다고 가정하자.
\[v'=\frac{mv_0}{bt+m}\]
여기서 차원을 분석해보면 다음과 같다.
\[[v']=\left[\frac{mv_0}{bt+m}\right]=\frac{[MLT^{-1}]}{[ML^{-1}][T]+[M]}\]
분모에서 더하는 두 값의 차원이 \(MLT^{-1}\), \(M\)으로 불일치가 일어나는 것을 알 수 있고, 따라서 중간에 계산 실수를 했음을 알 수 있다.

\subsection{차원 분석의 한계점}
차원 분석도 완벽하게 모든 경우를 커버하지는 않는다. 대표적인 예로 운동에너지 정의에 들어가는 \(1/2\)와 같은 무차원 상수가 빠졌는지는 알 수 없고, \(m=1\un{kg}\)과 같이 직접 변수에 값을 대입한 경우에도 따로 단위를 붙이지 않는 이상 잘못된 결과를 줄 수 있기 때문에, 문자로만 된 식에 대해 답이 `적당히 맞는지' 정도만 확인할 수 있다.

여담으로, 가끔 차원 분석이 저퀼리티 고등학교 물리 문제를 푸는데 쓰는 `꼼수'로 소개되기도 하는데, 조금이라도 생각이 있는 출제자는 선지의 차원을 모두 맞춰놓기 때문에 이 또한 한계점이라 할 수 있다.
\end{document}
