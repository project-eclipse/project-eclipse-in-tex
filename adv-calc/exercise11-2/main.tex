\documentclass{scrartcl}
\usepackage[margin=0.75in]{geometry}
\usepackage{amsmath}
\usepackage{amssymb}
\usepackage{amsthm}
\usepackage{fontspec}
\usepackage{graphicx}
\usepackage{hyperref}
\usepackage{mathrsfs}
\usepackage{mathtools}
\usepackage{pgf,tikz,pgfplots}
\usepackage{wrapfig}

\hypersetup{
    colorlinks=true,
    linkcolor=black,
    urlcolor=blue,
}
\pgfplotsset{compat=1.15}
\usetikzlibrary{arrows}
\newtheorem{theorem}{Theorem}

\setmainfont{Noto Serif CJK KR}
\setsansfont{Noto Sans CJK KR}
\newfontfamily\sectionfont{Noto Sans CJK KR Medium}
\addtokomafont{section}{\sectionfont\mdseries}
\addtokomafont{subsection}{\sectionfont\mdseries}
\addtokomafont{subsubsection}{\sectionfont\mdseries}
\addtokomafont{paragraph}{\sectionfont\mdseries}
\addtokomafont{title}{\sectionfont\mdseries}
\addtokomafont{author}{\sectionfont}
\title{11장 2절 연습문제 풀이}
\author{Project Eclipse}
\date{}

\newcommand{\Lim}[1]{\lim_{#1\to\infty}}
\newcommand{\para}{\mathbin{\!/\mkern-5mu/\!}}
\newcommand{\Seg}[1]{\overline{#1}}
\newcommand{\Line}[1]{\overleftrightarrow{#1}}
\newcommand{\Ray}[1]{\overrightarrow{#1}}
\newcommand{\infsum}[1]{\sum^\infty_{#1}}
\newcommand{\mat}[3]{\begin{pmatrix}
#1_{11} & #1_{12} & \cdots & #1_{1#3} \\
#1_{21} & #1_{22} & \cdots & #1_{2#3} \\
\vdots & \vdots & \ddots & \vdots \\
#1_{#2 1} & #1_{#2 2} & \cdots & #1_{#2#3} \\
\end{pmatrix}}
\newcommand{\un}[1]{\ensuremath{\ \mathrm{#1}}}

\begin{document}
\maketitle

\section{1번: 편미분 교환법칙}
그냥 무식하게 계산하면 된다. 1번은 다음과 같다.
\[D_1D_2(2x^2y^3-x^3y^5)=D_1(6x^2y^2-5x^3y^4)=12xy^2-15x^2y^4\]
\[D_2D_1(2x^2y^3-x^3y^5)=D_2(4xy^3-3x^2y^5)=12xy^2-15x^2y^4\]
2번은 다음과 같이 계산하면 된다.\footnote{임지안이 계산 과정 쓰는 나를 보고 ``이거 trivial하잖아''라고 했다.}
\[D_1D_2(\cos^7 x-\sin^2 y)=D_1(-2\sin y\cos y)=0\]
\[D_2D_1(\cos^7 x-\sin^2 y)=D_1(-7\sin x\cos^6 x)=0\]
3번도 비슷하게 계산하자.
\[D_1D_2(\arctan (xy))=D_1\left(\frac{x}{1+x^2y^2}\right)=\frac{1-x^2y^2}{(1+x^2y^2)^2}\]
\[D_2D_1(\arctan (xy))=D_2\left(\frac{y}{1+x^2y^2}\right)=\frac{1-x^2y^2}{(1+x^2y^2)^2}\]
4번도 다를 것은 없다. 물론 그대로 계산하는 것은 많은 근성을 필요로 하고, 결정적으로 힘들다. 로그를 쪼개서 계산하자.
\[D_1D_2(\log(x-1)-\log(y-1))=D_1\left(-\frac{1}{y-1}\right)=0\]
\[D_2D_1(\log(x-1)-\log(y-1))=D_2\left(\frac{1}{x-1}\right)=0\]

\section{2번: 모든 이계 편도함수}
3
\end{document}
