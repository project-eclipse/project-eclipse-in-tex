\documentclass{scrartcl}
\usepackage[left=2cm, right=2cm, top=2cm, bottom=2cm]{geometry}
\usepackage{amsmath}
\usepackage{amssymb}
\usepackage{amsthm}
\usepackage{blindtext}
\usepackage{datetime}
\usepackage{float}
\usepackage{fontspec}
\usepackage{graphicx}
\usepackage{hyperref}
\usepackage{mathrsfs}
\usepackage{mathtools}
\usepackage{pgf,tikz,pgfplots}
\usepackage{wrapfig}

\usepackage{blindtext}

\usepackage[headsepline]{scrlayer-scrpage}
\clearpairofpagestyles
\ohead{\thepage}
\ihead{Last compiled on: \today, \currenttime}

\hypersetup{
    colorlinks=true,
    linkcolor=black,
    urlcolor=blue,
}
\pgfplotsset{compat=1.15}
\usetikzlibrary{arrows}
\newtheorem{theorem}{Theorem}

\setmainfont{Noto Serif CJK KR}
\setsansfont{NanumSquare}
\newfontfamily\sectionfont{NanumSquare Bold}
\addtokomafont{section}{\sectionfont\mdseries}
\addtokomafont{subsection}{\sectionfont\mdseries}
\addtokomafont{subsubsection}{\sectionfont\mdseries}
\addtokomafont{paragraph}{\sectionfont\mdseries}
\addtokomafont{title}{\sectionfont\mdseries}
\addtokomafont{author}{\sectionfont}
\addtokomafont{date}{\sectionfont}
\title{곡선의 길이 연습문제 풀이}
\author{Project Eclipse (손량)}
\date{Last compiled on \today}

\newcommand{\Lim}[1]{\lim_{#1\to\infty}}
\newcommand{\para}{\mathbin{\!/\mkern-5mu/\!}}
\newcommand{\Seg}[1]{\overline{#1}}
\newcommand{\Line}[1]{\overleftrightarrow{#1}}
\newcommand{\Ray}[1]{\overrightarrow{#1}}
\newcommand{\infsum}[1]{\sum^\infty_{#1}}
\newcommand{\mat}[3]{\begin{pmatrix}
#1_{11} & #1_{12} & \cdots & #1_{1#3} \\
#1_{21} & #1_{22} & \cdots & #1_{2#3} \\
\vdots & \vdots & \ddots & \vdots \\
#1_{#2 1} & #1_{#2 2} & \cdots & #1_{#2#3} \\
\end{pmatrix}}
\newcommand{\un}[1]{\ensuremath{\ \mathrm{#1}}}

\begin{document}
\maketitle

\section{그냥 적분}
주어진 곡선을 한번 미분하고 속력을 구하면 다음과 같다.
\[X'(t)=(\cos t-t\sin t, \sin t+t\cos t, 1),\,|X'(t)|=\sqrt{t^2+2}\]
적분식을 세우면 다음과 같다.
\begin{align}\label{prob1-integration}
\int^{\sqrt{2}\sinh a}_0 \sqrt{t^2+2}\,dt
\end{align}
선생님께서는 그냥 울프람에 돌리거나 적분표를 보라고 하셨지만, 하라는 대로만 하면 재미없기 때문에 한번 적분을 해 보자.\footnote{물론 명분은 없다. 고등 미적분에 나올만한 적분도 아니기 때문이다. 명분 없는 수학이 싫으면 울프람에 넣어보고 2번으로 넘어가자.} \(t=\sqrt{2}\sinh u\)라고 치환한다고 하자. 이때 다음이 성립한다.
\[\frac{dt}{du}=\sqrt{2}\cosh u,\,dt=\sqrt{2}\cosh u\,du\]
따라서 (\ref{prob1-integration})의 식은 다음과 같이 적을 수 있다.
\begin{align}\label{prob1-substituted}
\int^a_0 \sqrt{2\sinh^2 u + 2}\cdot\sqrt{2}\cosh u\,du
\end{align}
기억이 잘 안나지만 1학기 때 다음이 성립함을 배웠다.
\[\cosh^2 x-\sinh^2 x=1\]
따라서 (\ref{prob1-substituted})의 식은 다음과 같이 쓸 수 있다.
\begin{align}\label{prob1-coshsquare}
\int^a_0 2\cosh^2 u\,du
\end{align}
\(\cosh x\)의 정의를 떠올려 보면
\[\cosh x=\frac{e^x+e^{-x}}{2}\]
이를 이용하면 (\ref{prob1-coshsquare})의 적분은 쉽게 해결된다.
\begin{align*}
2\int^a_0 \left( \frac{e^u+e^{-u}}{2} \right)^2\,du=&\frac{1}{2}\int^a_0 \left(e^{2u}+2+e^{-2u}\right)\,du \\
=&\frac{1}{2}\left[ \frac{1}{2}e^{2u}+2x-\frac{1}{2}e^{-2u} \right]^a_0 \\
=&\frac{e^{2a}+4a-e^{-2a}}{4}
\end{align*}

\section{포물선의 길이}
수업시간에 이미 한 관계로 그냥 넘어간다.

\section{아스트로이드의 길이}
\(y\geq0\)인 부분의 곡선은 다음과 같고,
\[X(t)=(t^3, (1 - t^2)^\frac{3}{2})\]
\(y\leq0\)인 부분의 곡선은 다음과 같이 나타낼 수 있다.
\[Y(t)=(t^3, -(1 - t^2)^\frac{3}{2})\]
\(-1\leq t\leq 1\)에서 \(X(t)\)와 \(Y(t)\)는 길이가 같으므로, \(X(t)\)의 길이를 구하고 2배하면 된다. 미분을 하고 속력을 구하면 다음과 같다.
\[X'(t) = \left(3t^2, \frac{3}{2}(1-t^2)^\frac{1}{2}\cdot(-2t)\right)=(3t^2, -3t\sqrt{1 - t^2}),\,|X'(t)|=\sqrt{9t^4+9t^2(1-t^2)}=\sqrt{9t^2}=|3t|\]
적분을 하면\footnote{말만 적분이지 삼각형 넓이 구하기이다.}
\[2\int^1_{-1} |3t|\,dt=6\]
원하는 길이를 구한다. 이제 원의 지름 비율을 구해야 하는데, 이 문제를 `정석적으로' 극좌표계를 통해 풀었다면 쉽게 나왔겠지만, 애석하게도 직교좌표를 기준으로 풀었기 때문에 다르게 구해야 한다. 말만 이렇지 실제로는 간단하다. 곡선은 점 \(1, 0)\)을 지나고, 그림에서 보면 알듯 이를 통해 큰 원의 반지름이 1임을 알 수 있다. 또한 곡선에서 원점과 가장 가까운 점은 직선 \(y=x\) 위에 있는데, 이 점의 좌표는 \(\left(1/2\sqrt{2}, 1/2\sqrt{2}\right)\)이다. 이 점과 원점의 거리는 \(1/2\)이고, 이는 큰 원의 반지름에서 작은 원의 지름을 뺀 값과 같다. 즉 작은 원의 반지름은 \(1/4\)이다. 구하는 비는 \(4:1\). 그림에 나온 곡선은 아스트로이드라고 불리는데, 이는 곡선의 모양이 일반적인 사람이 생각하는\footnote{물리학을 조금이라도 아는 사람은 별이 구형일 수밖에 없다고 주장하겠지만, 수학자들이 이미 이름을 정해버렸다...} 별의 모양과 비슷해서인듯 하다.

\section{곡선의 길이와 별 상관없는 풀이}
곡선의 길이를 다루는 부분의 연습문제라 곡선의 길이를 구하는 방법으로 문제를 풀려고 시도했지만, 그렇게 하지 못했고 그냥 `중학교 수준'의 수학으로 문제를 풀이하였다. 우선 자를 때의 몇 가지 가정이 필요한데, 사인곡선 형태의 모양이 자른 결과로 나오려면 자르는 평면이 원기둥의 밑면 둘 중 하나만 지나야 한다. 그렇지 않으면 타원이 나오기 때문이다. 이러한 가정 하에, 평면이 지나간 밑면의 점들을 그림과 같이 잡자.

\begin{figure}[H]
\centering
\definecolor{qqwuqq}{rgb}{0,0.39215686274509803,0}
\definecolor{xdxdff}{rgb}{0.49019607843137253,0.49019607843137253,1}
\definecolor{uuuuuu}{rgb}{0.26666666666666666,0.26666666666666666,0.26666666666666666}
\begin{tikzpicture}[line cap=round,line join=round,>=triangle 45,x=0.5cm,y=0.5cm]
\clip(-6,-6) rectangle (6,6);
\draw [shift={(0,0)},line width=1pt,color=qqwuqq,fill=qqwuqq,fill opacity=0.10000000149011612] (0,0) -- (66.4100098292567:0.6) arc (66.4100098292567:90:0.6) -- cycle;
\draw [shift={(0,0)},line width=1pt,color=qqwuqq,fill=qqwuqq,fill opacity=0.10000000149011612] (0,0) -- (90:0.6) arc (90:143.13010235415598:0.6) -- cycle;
\draw [line width=1pt] (0,0) circle (5);
\draw [line width=1pt] (-4,3)-- (4,3);
\draw [line width=1pt] (-4,3)-- (0,0);
\draw [line width=1pt] (0,0)-- (4,3);
\draw [line width=1pt,dash pattern=on 2pt off 2pt] (0,5)-- (0,0);
\draw [line width=1pt] (0,0)-- (2.000944668878205,4.5821632917310895);
\begin{scriptsize}
\draw [fill=uuuuuu] (0,0) circle (2pt);
\draw[color=uuuuuu] (0,-0.5) node {$A$};
\draw [fill=uuuuuu] (-4,3) circle (2.5pt);
\draw[color=uuuuuu] (-4.2,3.6) node {$B$};
\draw [fill=uuuuuu] (4,3) circle (2.5pt);
\draw[color=uuuuuu] (4.2,3.6) node {$C$};
\draw [fill=uuuuuu] (0,5) circle (2pt);
\draw[color=uuuuuu] (0.16,5.5) node {$D$};
\draw [fill=xdxdff] (2.000944668878205,4.5821632917310895) circle (2.5pt);
\draw[color=xdxdff] (2.16,5.1) node {$P'$};
\draw[color=qqwuqq] (0.2,0.85) node {$t$};
\draw[color=qqwuqq] (-0.35,0.8) node {$\phi$};
\end{scriptsize}
\end{tikzpicture}

\caption{밑면}
\label{fig:cylinder-circularplane}
\end{figure}
Figure \ref{fig:cylinder-circularplane}에서는 \(\Seg{BC}\)를 평면과 밑면의 교선으로 두었고, \(A\)를 원의 중심으로 두었다. 또한 호 \(BC\)를 이등분하는 점은 \(D\)로 잡았고, 호의 중심각의 반을 \(\phi\)로 잡았다. 또한, 매개화를 위해 \(\Seg{AD}\)와 \(t\)의 각도를 이루는 점 \(P'\)을 잡았다. \(P'\)에서 \(\Seg{AD}\)에 내린 수선의 발을 \(H\)라고 하면, 원기둥의 밑면 반지름이 1일때 다음과 같다.
\[\Seg{AH}=\cos t\]
잘려나온 종이를 Figure \ref{fig:sinusoid}에서처럼 좌표평면에 둘 수 있다. 단면이 밑면과 이루는 각을 \(\theta\)라고 하자. 이때 \(P\)의 좌표는 \((t, \tan\theta \cos t)\)로 나타낼 수 있으므로, 사인곡선임이 증명된다.

\begin{figure}[H]
\centering
\definecolor{uuuuuu}{rgb}{0.26666666666666666,0.26666666666666666,0.26666666666666666}
\definecolor{xdxdff}{rgb}{0.49019607843137253,0.49019607843137253,1}
\definecolor{qqwuqq}{rgb}{0,0.39215686274509803,0}
\begin{tikzpicture}[line cap=round,line join=round,>=triangle 45,x=1cm,y=1cm]
\begin{axis}[
x=1.5cm,y=1.5cm,
axis lines=middle,
xmin=-2.5,
xmax=2.5,
ymin=-0.5,
ymax=1.5,
ticks=none,]
\clip(-2.5,-0.5) rectangle (2.5,1.5);
\draw[line width=1.5pt,color=qqwuqq,smooth,samples=100,domain=-1.58:1.58] plot(\x,{cos(((\x))*180/pi)});
\draw [line width=1pt,dash pattern=on 2pt off 2pt] (0.9,0.6216099682706644)-- (0.9,0);
\begin{scriptsize}
\draw[color=uuuuuu] (-1.8,0.2) node {$-\phi$};
\draw[color=uuuuuu] (1.7,0.2) node {$\phi$};
\draw [fill=xdxdff] (0.9,0.6216099682706644) circle (2pt);
\draw[color=xdxdff] (1,0.8) node {$P$};
\draw [fill=uuuuuu] (0.9,0) circle (2pt);
\draw[color=uuuuuu] (1,-0.3) node {$P'$};
\end{scriptsize}
\end{axis}
\end{tikzpicture}

\caption{사인곡선}
\label{fig:sinusoid}
\end{figure}

\section{토리첼리의 정리}
곡선을 한번 미분하면 속도벡터를 얻는다.
\[X'(\theta)=r_0e^{k\theta}(k\cos\theta-\sin\theta,k\sin\theta+\cos\theta)\]
\(S\)는 곡선 위의 점 \(T\)에서 그은 접선 위에 있고, \(\Ray{OT}\)와 \(\Ray{OS}\)는 수직이므로 적절한 \(s\)에 대해 다음이 성립한다.
\begin{align}\label{prob2-innerprod}
\Ray{OT}\cdot\Ray{OS}=X(\theta)\cdot(X(\theta)+sX'(\theta))=0
\end{align}
(\ref{prob2-innerprod})의 식은 직접 대입하여 다음과 같이 쓸수도 있다. 스칼라는 내적 결과에 영향을 주지 않으므로 계산을 간단하게 하기 위해 미리 제외하고 계산했다.
\begin{align*}
&(\cos\theta, \sin\theta)\cdot((\cos\theta, \sin\theta)+s(k\cos\theta-\sin\theta, k\sin\theta+\cos\theta)) \\
&=1+(\cos\theta, \sin\theta)\cdot s(k\cos\theta-\sin\theta, k\sin\theta+\cos\theta) \\
&=1+sk\cos^2\theta-s\cos\theta\sin\theta+sk\sin^2\theta+s\sin\theta\cos\theta \\
&=1+sk=0
\end{align*}
즉, \(\Ray{OS}=X(\theta)-k^{-1}X'(\theta)\)임을 알 수 있다. \(\Seg{TS}\)를 계산하면 다음과 같다.
\begin{align*}
\Seg{TS}&=\left|2r_0e^{k\theta}(\cos\theta,\sin\theta)-k^{-1}r_0e^{k\theta}(k\cos\theta-\sin\theta,k\sin\theta+\cos\theta)\right| \\
&=\left|2r_0e^{k\theta}(\cos\theta,\sin\theta)-r_0e^{k\theta}(\cos\theta-k^{-1}\sin\theta,\sin\theta+k^{-1}\cos\theta)\right| \\
&=\left|r_0e^{k\theta}(\cos\theta,\sin\theta)-r_0e^{k\theta}(-k^{-1}\sin\theta,k^{-1}\cos\theta)\right| \\
&=r_0e^{k\theta}\left|(\cos\theta+k^{-1}\sin\theta,\sin\theta-k^{-1}\cos\theta)\right| \\
&=r_0e^{k\theta}\sqrt{1+\frac{1}{k^2}}
\end{align*}
한편, 곡선의 길이는 다음과 같이 구할 수 있고, \(\Seg{TS}\)와 같음을 알 수 있다.
\begin{align*}
\lim_{x\to-\infty}\int^\theta_x \sqrt{r(\phi)^2+r'(\phi)^2}\,d\phi&=\lim_{x\to-\infty}\int^\theta_x \sqrt{{r_0}^2e^{2k\phi}+k^2{r_0}^2e^{2k\phi}}\,d\phi \\
&=r_0\sqrt{1+k^2}\lim_{x\to-\infty}\int^\theta_x e^{k\phi}\,d\phi \\
&=r_0\sqrt{1+k^2}\lim_{x\to-\infty}\left[\frac{1}{k}e^{k\phi}\right]^\theta_x \\
&=r_0e^{k\theta}\frac{\sqrt{1+k^2}}{k}=r_0e^{k\theta}\sqrt{1+\frac{1}{k^2}}
\end{align*}
또한, 정의에 따라 \(\Seg{OT}\)의 길이는 \(r_0e^{k\theta}\)이므로 곡선의 길이와 상수 배 차이남을 알 수 있다. 이 문제에서는 아르키메데스 나선이 등각곡선임도 증명된다. 즉, \(\theta\)에 상관없이 \(\Ray{OT}\)와 \(\Ray{TS}\)의 사잇각이 일정하다는 것이다.

\section{아르키메데스 와선의 길이}
수업시간에 이미 한 관계로 그냥 넘어간다.

\section{하위헌스의 시계}
문제를 풀기 전에 우선 몇가지 가정을 할 것이다. 우선 시계 양옆에 있는 사이클로이드는 원래 곡선의 절반을 자른 것이고 (즉, \(0\leq t\leq\pi\)이다.), 실의 길이는 4라는 것이다. 실이 고정된 지점의 좌표를 원점으로 놓고, 시계의 오른쪽 부분의 사이클로이드를 다음과 같이 정의하자.
\[X(t)=(t-\sin t, \cos t-1)\]
사이클로이드의 속력을 구하는 과정은 책에도 설명되어 있으니 생략한다.
\[|X'(t)|=\left|2\sin\frac{t}{2}\right|=2\sin\frac{t}{2}\]
추가 그리는 궤적의 오른쪽 부분을 매개화된 곡선 \(Y(t)\)로 나타낸다고 하자. 실이 원점부터 \(X(t)\)까지 사이클로이드와 맞닿아 있다고 하면, 다음과 같이 쓸 수 있다.
\[Y(t)=X(t)+X'(t)\int^t_0 |X'(t)|\,dt\]
대입을 해 보면 다음과 같은 식을 얻는다.
\begin{align*}
Y(t)&=(t-\sin t, \cos t-1)+\frac{(1-\cos t, -\sin t)}{2\sin\frac{t}{2}}\cdot4\cos\frac{t}{2} \\
&=(t-\sin t, \cos t-1)+\frac{(2\sin^2\frac{t}{2}, -\sin\frac{t}{2}\cos\frac{t}{2})}{2\sin\frac{t}{2}}\cdot4\cos\frac{t}{2} \\
&=(t-\sin t, \cos t-1)+4\cos\frac{t}{2}\left(\sin\frac{t}{2}, -\cos\frac{t}{2}\right) \\
&=(t-\sin t, \cos t-1)+(2\sin t, -2\cos t-2) \\
&=(t+\sin t, -\cos t-3) \\
&=(\tau-\pi-\sin\tau, \cos\tau-3)=(\tau-\sin\tau, \cos\tau-1)-(\pi,-2)
\end{align*}
마지막 식에서 \(\tau=\pi+t\)로 잡았다. 궤적은 좌우 대칭이므로 원하는 결론을 얻는다.

\section{극좌표계의 곡선 길이 구하기}
\begin{align}
\int^2_1 \sqrt{9\theta^4+36\theta^2}\,d\theta&=3\int^2_1 \theta\sqrt{\theta^2+4}\,d\theta \nonumber \\
\label{prob8-first}&=3\int^{2\sqrt{2}}_{\sqrt{5}} t^2\,dt=\left[t^3\right]^{2\sqrt{2}}_{\sqrt{5}}=16\sqrt{2}-25\sqrt{5} \\
\int^2_1 \sqrt{e^{-8\theta}+16e^{-8\theta}}\,d\theta&=\sqrt{17}\int^2_1 e^{-4\theta}\,d\theta=\sqrt{17}\left[-\frac{1}{4}e^{-4\theta}\right]^2_1 \nonumber \\
&=\sqrt{17}\left( -\frac{1}{4}e^{-8}+\frac{1}{4}e^{-4} \right) \nonumber \\
\int^{\sqrt{3}}_1 \sqrt{\frac{4}{\theta^2}+\frac{4}{\theta^4}}\,d\theta&=2\int^{\sqrt{3}}_1\frac{\sqrt{\theta^2+1}}{\theta^2}\,d\theta \nonumber \\
\label{prob8-third}&=2\left[-\log\left( \sqrt{1+\theta^2}-\theta \right)-\frac{\sqrt{1+\theta^2}}{\theta}\right]^{\sqrt{3}}_1 \\
\int^\pi_0 \sqrt{\sin^4\frac{\theta}{2}+\sin^2\frac{\theta}{2}\cos^2\frac{\theta}{2}}\,d\theta&=\int^\pi_0 \sin\frac{\theta}{2}\sqrt{\sin^2\frac{\theta}{2}+\cos^2\frac{\theta}{2}}\,d\theta \nonumber \\
&=\int^\pi_0 \sin\frac{\theta}{2}\,d\theta=\left[-2\cos\frac{\theta}{2}\right]^\pi_0=2 \nonumber
\end{align}
(\ref{prob8-first})에서 \(t=\sqrt{\theta^2+4}\)라고 치환했다. (\ref{prob8-third})에서 한 적분은 적분표에 나와있다.

\section{로그의 길이}
곡선을 다음과 같이 놓자.
\[X(t)=(t, \log t)\qquad a\leq t\leq b\]
미분을 한번 해서 속도와 속력을 구하면 다음과 같다.
\[X'(t)=\left( 1, \frac{1}{t} \right),\,|X'(t)|=\sqrt{1+\frac{1}{t^2}}\]
늘 하듯 적분식을 세우면 다음과 같다.
\begin{align}\label{prob9-integration}
\int^b_a \sqrt{1+\frac{1}{t^2}}\,dt=\int^b_a \frac{\sqrt{1+t^2}}{t}\,dt
\end{align}
그냥 적분표를 보고 싶은 생각도 들지만, 한번 적분해보자.\footnote{이 적분은 고등학교 미적분에도 나올만 하기 때문에 명분은 있다.} \(t=\tan u\)로 치환하고, \(\alpha=\arctan a,\,\beta=\arctan b\)라고 두면 (\ref{prob9-integration})의 식을 다음과 같이 쓸 수 있다.
\begin{align}
\label{prob9-sec-tan-relation}\int^\beta_\alpha \frac{\sqrt{1+\tan^2 u}}{\tan u}\cdot\sec^2u\,du&=\int^\beta_\alpha \frac{\sec u}{\tan u}\cdot\sec^2u\,du \\
&=\int^\beta_\alpha \csc u\sec^2 u\,du \nonumber \\
&=\int^\beta_\alpha (\csc u+\sec u\tan u)\,du \nonumber \\
&=\left[\sec u\right]^\beta_\alpha+\int^\beta_\alpha \csc u\,du \nonumber \\
&=\left[\sec u\right]^\beta_\alpha+[-\log\left|\csc u+\cot u\right|]^\beta_\alpha \nonumber
\end{align}
(\ref{prob9-sec-tan-relation})에서 \(\tan^2 x+1=\sec^2 x\)라는 관계를 사용했다. 따라서 구하는 길이는 다음과 같다.
\[\left(\sqrt{b^2+1}-\log\left|\frac{\sqrt{b^2+1}+1}{b}\right|\right)-\left(\sqrt{a^2+1}-\log\left|\frac{\sqrt{a^2+1}+1}{a}\right|\right)\]
\end{document}
