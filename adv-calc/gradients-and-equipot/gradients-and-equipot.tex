\documentclass{scrartcl}
\usepackage[margin=0.5in]{geometry}
\usepackage{amsmath}
\usepackage{amssymb}
\usepackage{amsthm}
\usepackage{fontspec}
\usepackage{graphicx}
\usepackage{hyperref}
\usepackage{mathrsfs}
\usepackage{mathtools}
\usepackage{pgf,tikz,pgfplots}
\usepackage{pstricks-add}
\usepackage{pst-3dplot,pst-func,pst-math}
\usepackage{wrapfig}

\hypersetup{
    colorlinks=true,
    linkcolor=black,
    urlcolor=blue,
}
\pgfplotsset{compat=1.15}
\usetikzlibrary{arrows}
\newtheorem{theorem}{Theorem}

\setmainfont{Noto Serif CJK KR}
\setsansfont{Noto Sans CJK KR}
\newfontfamily\sectionfont{Noto Sans CJK KR Medium}
\addtokomafont{section}{\sectionfont\mdseries}
\addtokomafont{subsection}{\sectionfont\mdseries}
\addtokomafont{subsubsection}{\sectionfont\mdseries}
\addtokomafont{paragraph}{\sectionfont\mdseries}
\addtokomafont{title}{\sectionfont\mdseries}
\addtokomafont{author}{\sectionfont}
\title{등위면과 그래디언트의 기하학적 의미}
\author{Project Eclipse (손량)}
\date{}

\newcommand{\Lim}[1]{\lim_{#1\to\infty}}
\newcommand{\para}{\mathbin{\!/\mkern-5mu/\!}}
\newcommand{\Seg}[1]{\overline{#1}}
\newcommand{\Line}[1]{\overleftrightarrow{#1}}
\newcommand{\Ray}[1]{\overrightarrow{#1}}
\newcommand{\infsum}[1]{\sum^\infty_{#1}}
\newcommand{\mat}[3]{\begin{pmatrix}
#1_{11} & #1_{12} & \cdots & #1_{1#3} \\
#1_{21} & #1_{22} & \cdots & #1_{2#3} \\
\vdots & \vdots & \ddots & \vdots \\
#1_{#2 1} & #1_{#2 2} & \cdots & #1_{#2#3} \\
\end{pmatrix}}
\newcommand{\grad}{\text{grad}}
\newcommand{\un}[1]{\ensuremath{\ \mathrm{#1}}}

\begin{document}
\maketitle

미분계수의 의미를 그래프의 기울기라고 알고 있고, \(\grad\)를 이러한 이유로 접평면과 상관 있다고 추측하다가 정작 의미를 이해하는 데에는 어려움을 겪었다.\footnote{어려움을 겪지 않았다면 굳이 이걸 읽으려고 할 필요는 없을 것 같다.} 여기에서는 \(\grad\)가 실제로 무엇인지 살펴볼 것이다.

\section{'그래프'와 '등위면'}
고등수학에서 보통 좌표평면 위에 곡선이 그려져 있으면 이를 `그래프'라고 부른다. 지금까지는 크게 신경 쓰지 않았지만, 우리가 `그래프'라고 부르는 것은 크게 두 가지로 나눌 수 있다. \textbf{함수의 그래프}와 \textbf{음함수의 그래프}이다. 이들 각각을 보여주는 대표적인 예로 함수 \(y=x^2\)과 원의 방정식 \(x^2+y^2=1\)을 그린 것을 생각할 수 있을 것이다.
\begin{wrapfigure}[13]{l}{5cm}
  \centering
  \documentclass[10pt]{article}
\usepackage{pstricks-add}
\pagestyle{empty}
\usepackage{pst-func}
\begin{document}
\newrgbcolor{ududff}{0.30196078431372547 0.30196078431372547 1}
\psset{xunit=1cm,yunit=1cm,algebraic=true,dimen=middle,dotstyle=o,dotsize=5pt 0,linewidth=1pt,arrowsize=2pt 2,arrowinset=0.25}
\begin{pspicture*}(-3.5,-3.5)(3.5,3.5)
\multips(0,-6)(0,1){14}{\psline[linestyle=dashed,linecap=1,dash=1.5pt 1.5pt,linewidth=0.4pt,linecolor=lightgray]{c-c}(-8.92,0)(8.92,0)}
\multips(-8,0)(1,0){18}{\psline[linestyle=dashed,linecap=1,dash=1.5pt 1.5pt,linewidth=0.4pt,linecolor=lightgray]{c-c}(0,-6.92)(0,6.92)}
\psaxes[labelFontSize=\scriptstyle,xAxis=true,yAxis=true,Dx=1,Dy=1,ticksize=-2pt 0,subticks=2]{->}(0,0)(-3.5,-3.5)(3.5,3.5)
\psplotImp[linewidth=1.5pt,stepFactor=0.5](-10,-8)(9,7){1*y^3-3*x^1*y^1+1*x^3}
\psline[linewidth=1pt]{->}(-3,-3)(3,3)
\psline[linewidth=1pt]{->}(3,-3)(-3,3)
\psplot[linewidth=1.5pt,linecolor=blue]{-3.92}{3.92}{(-0--2*x)/1}
\begin{scriptsize}
\rput[bl](3.08,3.2){$x'$}
\rput[bl](-3.1,3.2){$y'$}
\rput[bl](0.5,3){\blue{$y'=tx'$}}
\end{scriptsize}
\end{pspicture*}
\end{document}

\end{wrapfigure}
음함수는 정의에 따라, 어떤 다변수함수의 0-등위면으로 이해할 수 있다. 예를 들어, 원의 방정식 \(x^2+y^2=1\)은 함수
\[f(x,y)=x^2+y^2-1\]
의 0-등위면으로 생각할 수 있다. 여기서 중요한 점은, 굳이 음함수가 아니여도 그래프를 등위면으로 생각할 수 있다는 것이다. 예를 들어, 앞서 말한 함수 \(y=x^2\)은 다음과 같은 2변수함수의 0-등위면으로 생각할 수 있다. 다르게 말하면, 음함수의 그래프는 \textbf{등위면의 그래프}라고 할 수 있다는 것이다.\footnote{음함수의 그래프와 등위면의 그래프는 설명을 위해 지어낸 말이기 때문에 공식적인 수학적 주장에서는 clarification이 필요하다.}
\[f(x,y)=x^2-y\]
이러한 변환은 다변수\textbf{함수의 그래프} 접평면을 구하는 데 사용될 것이다.

\section{그래디언트의 기하학적 의미}
미적분학 2+ 책에서는 \(\grad\)의 기하학적 의미를 다음과 같이 설명하고 있다.
\begin{theorem}
\(\grad f(P)\)는 점 \(P\)가 속하는 등위면 \(S:=\{X\in U|f(X)=f(P)\}\)에 수직이다. (\(\grad f(P)\neq\mathbf{0}\))
\end{theorem}
등위면에 수직인 벡터라는 것은 곧 점 \(P\)의 접평면과 수직이라는 것이고, 여기서 \(\grad f(P)\)가 등위면 \(f(X)=f(P)\)의 그래프의 접평면 법선벡터임을 알 수 있다. 즉, \(\grad f(P)\)는 \textbf{등위면의 그래프}의 접평면을 나타낸다. 이를 좀 더 명확히 하기 위해, 몇가지 예시를 살펴보자.

\subsection{2차원: 함수의 그래프}
간단한 함수 \(y=f(x)=x^2\)의 접평면을 구해 보자.\footnote{여기서 평면은 흔히 말하는 3차원의 `평평한 판'이 아니라 초평면을 의미한다. 이 글에서는 굳이 초평면과 평면을 구분하지 않을 것이다.} 이때 접평면은 접선의 모습으로 나타날 것이다. 여기에서, 그냥 \(\grad f(p)\)를 구하면 아무것도 할 수 없다. 왜냐하면 \(\grad f(p)=(2p)\)이고, 이를 법선벡터로 생각하면 다음과 같이 의미 없는 `평면의 방정식'이 나오기 때문이다.
\[2t\cdot(x-p)=0\]
이 `평면'에 있는 점은 오직 \((p,p^2)\)뿐이다. 어디서 잘못된 것일까?\\[1\baselineskip]
문제는 정리를 적절하지 않은 방법으로 사용했다는 것에 있다. 앞서 말했듯 \(\grad\)는 등위면의 접평면이라고 했다. 다르게 말하면 그래프 \(y=f(x)\)의 접평면을 구하기 위해서는 다음과 같이 정의한 함수의 \(\grad\)를 구해야 한다.
\[g(x,y)=f(x)-y\]
\(P=(x_0,y_0)\)이라고 잡고 계산을 해 보면 \(\grad g(P)=(2x_0,-1)\)이고, 접평면의 방정식은 따라서 다음과 같다.
\[(2x_0,-1)\cdot(x-x_0,y-y_0)=0\]
이를 전개하면 다음과 같은 식을 얻는다.
\[y=2x_0(x-x_0)+y_0\]
이는 고등수학에서 말하는 접선의 방정식과 같다.

\subsection{2차원: 등위면의 그래프}
데카르트 곡선 \(x^3+y^3-3xy=0\)의 접평면을 구해 보자. 우선 이 음함수를 다음 다변수함수의 0-등위면으로 생각할 수 있다.
\[f(x,y)=x^3+y^3-3xy\]
계산을 하면
\[\grad f(x_0,y_0)=3({x_0}^2-y_0,{y_0}^2-x_0)\]
따라서 접평면의 방정식은
\[3({x_0}^2-y_0,{y_0}^2-x_0)\cdot(x-x_0,y-y_0)=0\]

\subsection{3차원: 등위면의 그래프}
함수의 그래프는 간단하기 때문에, 수업 시간에 다룬 데카르트 곡선을 만드는 함수의 접평면을 구해 보자.
\[f(x,y)=x^3+y^3-3xy\]
\begin{center}
  \begin{pspicture}(-3.5, -3.5)(3.5, 5)
\psset{xunit=0.25cm,yunit=0.25cm,Alpha=50}
\psplotThreeD[xPlotpoints=1000,algebraic,linecolor=blue,plotstyle=line](-3,3)(-3,3){x^3+y^3-3xy}
\pstThreeDCoor[xMin=-10,xMax=10,yMin=-10,yMax=10,zMax=10,zMin=-10]
\end{pspicture}

\end{center}

\end{document}
