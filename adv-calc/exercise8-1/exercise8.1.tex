\documentclass{scrartcl}
\usepackage[margin=0.5in]{geometry}
\usepackage{amsmath}
\usepackage{amssymb}
\usepackage{amsthm}
\usepackage{fontspec}
\usepackage{graphicx}
\usepackage{hyperref}
\usepackage{mathrsfs}
\usepackage{mathtools}
\usepackage{pgf,tikz,pgfplots}

\hypersetup{
    colorlinks=true,
    linkcolor=black,
    urlcolor=blue,
}
\pgfplotsset{compat=1.15}
\usetikzlibrary{arrows}
\newtheorem{theorem}{Theorem}

\setmainfont{Noto Serif CJK KR}
\setsansfont{Noto Sans CJK KR}
\newfontfamily\sectionfont{Noto Sans CJK KR Medium}
\addtokomafont{section}{\sectionfont\mdseries}
\addtokomafont{subsection}{\sectionfont\mdseries}
\addtokomafont{subsubsection}{\sectionfont\mdseries}
\addtokomafont{paragraph}{\sectionfont\mdseries}
\addtokomafont{title}{\sectionfont\mdseries}
\addtokomafont{author}{\sectionfont}
\title{8장 1절 연습문제 풀이}
\author{Project Eclipse (19092 손량)}
\date{}

\newcommand{\Lim}[1]{\lim_{#1\to\infty}}
\newcommand{\para}{\mathbin{\!/\mkern-5mu/\!}}
\newcommand{\Seg}[1]{\overline{#1}}
\newcommand{\Line}[1]{\overleftrightarrow{#1}}
\newcommand{\Ray}[1]{\overrightarrow{#1}}
\newcommand{\infsum}[1]{\sum^\infty_{#1}}
\newcommand{\mat}[3]{\begin{pmatrix}
#1_{11} & #1_{12} & \cdots & #1_{1#3} \\
#1_{21} & #1_{22} & \cdots & #1_{2#3} \\
\vdots & \vdots & \ddots & \vdots \\
#1_{#2 1} & #1_{#2 2} & \cdots & #1_{#2#3} \\
\end{pmatrix}}
\newcommand{\un}[1]{\ensuremath{\ \mathrm{#1}}}

\begin{document}
\maketitle

\section{쓸데없는 이야기: 라플라스 전개}
예전 문제풀이에서도 다뤘지만, 벡터곱을 할 때 행렬식을 구할 일이 많으므로 잠깐 소개한다.\footnote{이미 알고 있거나, 이런 걸 볼때 그저 지건이 마렵다면 연습문제 풀이로 넘어가면 된다.}\footnote{사실 새로 치기 귀찮아서 예전에 쓴 TeX 코드 복붙했다.} 탐구문제에도 있었고, 이충훈 선생님께서 기하 시간에 외적을 구할 때에도 잠깐 보여 주셨지만 라플라스 전개는 행렬식을 쉽게\footnote{물론 쉽다는 것은 계산하는 순서가 덜 헷갈린다는 거지, 실제로 하는 곱셈의 횟수는 그냥 구할 때하고 비슷하다.} 계산하는 방법이다. 다음 행렬의 행렬식을 한 번 라플라스 전개로 계산해 보자.\footnote{이게 왜 가능한지는 \href{https://en.wikipedia.org/wiki/Laplace_expansion\#Proof}{증명}을 참고하자.}
\[
\begin{pmatrix}
1 & 2 & 3\\
4 & 5 & 6\\
7 & 8 & 9
\end{pmatrix}
\]
\paragraph{행/열 정하기} 우선 주어진 행렬에서 열이나 행을 하나 정해야 한다. 여기에서는 그냥 첫 번째 열을 고르자. 어떤 행이나 열을 고르든 계산의 결과는 같다.\footnote{이것은 숫자로만 이루어진 `일반적인' 행렬의 경우만 그렇고, 벡터곱을 구할 때 사용하는 행렬의 경우 무조건 단위벡터 \(\mathbf{e}_1\), \(\mathbf{e}_2\), \(\mathbf{e}_3\)가 포함된 행을 골라야 한다.}
\paragraph{행렬 분해하기} 앞서 고른 행이나 열의 각각의 항에 대해, 각각의 항들을 지나는 열과 행을 지우고, 남은 항들로 새로운 행렬을 만든다. $n$차 정사각행렬이라면, $n-1$차 정사각행렬이 $n$개 만들어질 것이다. 여기서 예로 든 정사각행렬로 보면, 파란색으로 되어 있는 항을 기준으로 분해할 때를 보면 다음과 같다. 빨간색으로 표시된 항들은 분해 과정에서 없어지는 항들이다.
\[
\begin{pmatrix}
\color{blue}1 & \color{red}2 & \color{red}3\\
\color{red}4 & 5 & 6\\
\color{red}7 & 8 & 9
\end{pmatrix}\longrightarrow\begin{pmatrix}
5 & 6\\
8 & 9
\end{pmatrix},
\begin{pmatrix}
\color{red}1 & 2 & 3\\
\color{blue}4 & \color{red}5 & \color{red}6\\
\color{red}7 & 8 & 9
\end{pmatrix}\longrightarrow\begin{pmatrix}
2 & 3\\
8 & 9
\end{pmatrix},
\begin{pmatrix}
\color{red}1 & 2 & 3\\
\color{red}4 & 5 & 6\\
\color{blue}7 & \color{red}8 & \color{red}9
\end{pmatrix}\longrightarrow\begin{pmatrix}
2 & 3\\
5 & 6
\end{pmatrix}
\]
\paragraph{행렬식 나눠서 계산하기} 분해된 각각의 행렬의 행렬식을 구하고, 행렬을 분해할 때 기준으로 잡은 항을 곱해준다.
\[
1\det\begin{pmatrix}
5 & 6\\
8 & 9
\end{pmatrix}=-3,
4\det\begin{pmatrix}
2 & 3\\
8 & 9
\end{pmatrix}=-24,
7\det\begin{pmatrix}
2 & 3\\
5 & 6
\end{pmatrix}=-21
\]
\paragraph{부호 붙여서 결과값 내기} 이제 각각의 결과값에 부호를 붙여서 더하면 된다. 기준으로 잡은 항이 $(i, j)$항일 때, 앞서 구한 행렬식의 값들에 $(-1)^{i+j}$를 곱해준다. 이제 행렬식의 값을 구할 수 있다.
\begin{align*}
\det\begin{pmatrix}
1 & 2 & 3\\
4 & 5 & 6\\
7 & 8 & 9
\end{pmatrix}&=1\det\begin{pmatrix}
5 & 6\\
8 & 9
\end{pmatrix}-4\det\begin{pmatrix}
2 & 3\\
8 & 9
\end{pmatrix}+7\det\begin{pmatrix}
2 & 3\\
5 & 6
\end{pmatrix}\\
&=-3+24-21=0
\end{align*}

\section{연습문제 풀이}
\subsection{3번: 직선 구하기}
우선 방향벡터를 구하자.
\[(1,2,3)\times(-3,-2,4)=\det\begin{pmatrix}\mathbf{e}_1 & \mathbf{e}_2 & \mathbf{e}_3\\
1 & 2 & 3 \\
-3 & -2 & 4
\end{pmatrix}=(14,13,4)
\]
직선을 \(X(t)\)라고 놓으면\footnote{배웠으니까 직선도 곡선이지? 라는 생각을 지금쯤 하고 있을 것이다.}, 다음과 같이 답을 구할 수 있다.
\[X(t)=t(14,13,4)+(2,1,-5)\]

\subsection{8번: 벡터곱과 일차독립}
일차독립이 기억 안나는 사람들도 있을텐데, 참고로 이 문제에서 일차독립을 다르게 말하면 방정식
\[t\mathbf{a}+u\mathbf{b}=\mathbf{0}\]
에서 자명하지 않은 해가 존재하지 않는다는 뜻이다. 즉, \(t=u=0\) 이외에는 해가 존재하지 않는다는 것이다.

\subsubsection{\((\text{선형독립})\longrightarrow\mathbf{a}\times\mathbf{b}\neq\mathbf{0}\) 증명하기}
귀류법을 써보자. \(t=\alpha\), \(u=\beta\)라는 자명하지 않은 해가 존재하고, 그러면서 벡터곱을 한 결과가 영벡터가 아니였다고 해 보자. 다음과 같이 경우를 나눌 수 있다.
\paragraph{\(\alpha\)와 \(\beta\) 모두 0이 아닌 경우}
이 경우 \(\mathbf{a}=-\beta\mathbf{b}/\alpha\)이므로, 벡터곱을 계산하면
\[\mathbf{a}\times\mathbf{b}=-\frac{\beta}{\alpha}\mathbf{b}\times\mathbf{b}=\mathbf{0}\]

\paragraph{\(\alpha\)와 \(\beta\) 둘 중 하나가 0인 경우}
간단하다. \(\mathbf{a}\)와 \(\mathbf{b}\) 둘 중 하나가 영벡터라는 뜻이므로 벡터곱을 하면 \(\mathbf{0}\)이 나온다. 이로써 처음에 한 가정이 모순임을 알 수 있고, 원하는 결과를 얻는다.

\subsubsection{\(\mathbf{a}\times\mathbf{b}\neq\mathbf{0}\longrightarrow(\text{선형독립})\) 증명하기}
대우를 취하면 \((\text{선형독립이 아님})\longrightarrow\mathbf{a}\times\mathbf{b}=\mathbf{0}\)인데, 이는 앞서 귀류법으로 증명한 것과 유사하므로 생략한다.

\subsection{9번: 그냥 계산하기 싫은 문제}
처음에 문제들을 보고 무식하게 계산하기는 싫다는 생각이 들었다. 계산하는 것을 최대한 미루기 위해 일단 명제들 간의 관계들을 살펴보았다. 우선, (a)와 (b)는 거의 같은 식이다. 식 (a)에서,
\[\mathbf{a}\times(\mathbf{b}\times\mathbf{c})=(\mathbf{a}\cdot\mathbf{c})\mathbf{b}-(\mathbf{a}\cdot\mathbf{b})\mathbf{c}\]
외적의 순서를 바꾸고 \(\mathbf{a}\)를 \(\mathbf{c}\)로, \(\mathbf{b}\)를 \(\mathbf{a}\)로, \(\mathbf{c}\)를 \(\mathbf{b}\)로 바꾸면 식 (b)를 얻는다.
\[(\mathbf{a}\times\mathbf{b})\times\mathbf{c}=(\mathbf{a}\cdot\mathbf{c})\mathbf{b}-(\mathbf{b}\cdot\mathbf{c})\mathbf{a}\]
그리고 (c)에서 좌변을 외적의 성질을 이용하여 변형하면 식 (a)를 이용할 수 있고, (a)가 참이면 (c)도 참임을 알 수 있다.
\[\mathbf{a}\cdot(\mathbf{b}\times(\mathbf{c}\times\mathbf{d}))=\mathbf{a}\cdot((\mathbf{b}\cdot\mathbf{d})\mathbf{c}-(\mathbf{b}\cdot\mathbf{c})\mathbf{d})=(\mathbf{a}\cdot\mathbf{c})(\mathbf{b}\cdot\mathbf{d})-(\mathbf{a}\cdot\mathbf{d})(\mathbf{b}\cdot\mathbf{c})\]
(d)도 (a)를 세 번 사용하면 된다.
\begin{align*}
&\mathbf{a}\times(\mathbf{b}\times\mathbf{c})+\mathbf{b}\times(\mathbf{c}\times\mathbf{a})+\mathbf{c}\times(\mathbf{a}\times\mathbf{b}) \\
&=((\mathbf{a}\cdot\mathbf{c})\mathbf{b}-(\mathbf{a}\cdot\mathbf{b})\mathbf{c})+((\mathbf{b}\cdot\mathbf{a})\mathbf{c}-(\mathbf{b}\cdot\mathbf{c})\mathbf{a})+((\mathbf{c}\cdot\mathbf{b})\mathbf{a}-(\mathbf{c}\cdot\mathbf{a})\mathbf{b})=\mathbf{0}
\end{align*}
따라서 (a)만 증명하면 됨을 알 수 있다. 무식하게 계산하는건 정말 할 것이 못 되기 때문에, 기하학적 의미를 생각해 보자. \(\mathbf{b}\times\mathbf{c}\)는 \(\mathbf{b}\)와 \(\mathbf{c}\) 모두에 대해 수직이고, \(\mathbf{a}\times(\mathbf{b}\times\mathbf{c})\)는 \(\mathbf{b}\times\mathbf{c}\)에 대해 수직이므로 \(\mathbf{b}\), \(\mathbf{c}\)와 같은 평면에 존재한다. 이 때문에 우변에 나온 것처럼 \(\mathbf{b}\)와 \(\mathbf{c}\)의 선형결합으로 쓸 수 있는 것처럼 쓸 수 있는 것이다. 여기에서, 서로 수직이고, 다음 관계를 만족하는 벡터 \(\mathbf{i}\), \(\mathbf{j}\), \(\mathbf{k}\)를 생각해 보자.
\[\mathbf{i}\times\mathbf{j}=\mathbf{k},\,\mathbf{j}\times\mathbf{k}=\mathbf{i},\,\mathbf{k}\times\mathbf{i}=\mathbf{j}\]
여기에서 \(\mathbf{b}=b\mathbf{i}\), \(\mathbf{c}=c_x\mathbf{i}+c_y\mathbf{j}\), \(\mathbf{a}=a_x\mathbf{i}+a_y\mathbf{j}+a_z\mathbf{k}\)라고 가정하자.\footnote{눈치챘겠지만 이 풀이 방법에서는 \(\mathbf{a}\), \(\mathbf{b}\), \(\mathbf{c}\)에 대해 좌표계를 정의하고 이를 이용하여 계산을 편리하게 한다. 임의로 좌표계를 만들어서 증명하는 것이 불편하다면 그람-슈미트 방법으로 주어진 벡터에서 기저를 뽑아내서 증명할 수도 있을 것이다.} 이제 직접 계산하는 것이 생각만큼 복잡하지 않을 것이다. 우선 \(\mathbf{b}\)와 \(\mathbf{c}\)의 벡터곱은 다음과 같다.\footnote{책의 각주에서 \(\mathbf{b}\), \(\mathbf{c}\)가 표준단위벡터인 경우만 증명해도 된다고 한 것은 여기에서 중복된 성분들이 벡터곱하면서 사라지기 때문일 것이다.}
\[\mathbf{b}\times\mathbf{c}=b_x\mathbf{i}\times(c_x\mathbf{i}+c_y\mathbf{j})=b_xc_y(\mathbf{i}\times\mathbf{j})=b_xc_y\mathbf{k}\]
전체 식은 다음과 같이 구할 수 있고, 원하는 결과를 얻는다.
\begin{align*}
\mathbf{a}\times(\mathbf{b}\times\mathbf{c})&=a_xb_xc_y(\mathbf{i}\times\mathbf{k})+a_yb_xc_y(\mathbf{j}\times\mathbf{k})=-a_xb_xc_y\mathbf{j}+a_yb_xc_y\mathbf{i} \\
&=-a_xb_xc_x\mathbf{i}-a_xb_xc_y\mathbf{j}+a_xb_xc_x\mathbf{i}+a_yb_xc_y\mathbf{i} \\
&=-a_xb_x(c_x\mathbf{i}+c_y\mathbf{j})+(a_xc_x+a_yc_y)b_x\mathbf{i} \\
&=-(\mathbf{a}\cdot\mathbf{b})\mathbf{c}+(\mathbf{a}\cdot\mathbf{c})\mathbf{b}=(\mathbf{a}\cdot\mathbf{c})\mathbf{b}-(\mathbf{a}\cdot\mathbf{b})\mathbf{c}
\end{align*}

\subsection{13번: 정사영과 넓이}
\(\mathbf{a}\)와 \(\mathbf{b}\)를 정사영하여 벡터 \(\mathbf{a}'\), \(\mathbf{b}'\)을 얻었고, \(\mathbf{a}\)와 \(\mathbf{b}\)가 이루는 평면의 법선벡터를 \(\mathbf{n}=\mathbf{a}\times\mathbf{b}\), 정사영한 평면의 법선벡터를 \(\mathbf{n}'=\mathbf{a}'\times\mathbf{b}'\)이라고 하자. 정사영한 평행사변형의 넓이는 외적을 이용하여 간단히 구할 수 있다.
\[S'=\left|\mathbf{a}'\times\mathbf{b}'\right|=\left|\mathbf{n}'\right|\]
한편, \(\mathbf{a}\)와 \(\mathbf{b}\)는 다음과 같이 나타낼 수 있다.
\[\mathbf{a}=\mathbf{a}'+\alpha\mathbf{n}',\,\mathbf{b}=\mathbf{b}'+\beta\mathbf{n}'\]
이를 이용하여 \(\mathbf{n}\)을 나타내면
\begin{align*}
\mathbf{n}&=\mathbf{a}\times\mathbf{b}=\left(\mathbf{a}'+\alpha\mathbf{n}'\right)\times\left(\mathbf{b}'+\beta\mathbf{n}'\right) \\
&=\mathbf{a}'\times\mathbf{b}'+\beta\mathbf{a}'\times\mathbf{n}'+\alpha\mathbf{b}'\times\mathbf{n}' \\
&=\mathbf{n}'+\beta\mathbf{n}'\times\mathbf{a}'+\alpha\mathbf{n}'\times\mathbf{b}'
\end{align*}
복잡해 보이지만 여기에서 \(\mathbf{n}\)과 \(\mathbf{n}'\)을 내적하면, \(\mathbf{n}'\)과 벡터곱한 항들은 모두 사라지므로 다음과 같다.
\[\mathbf{n}\cdot\mathbf{n}'=\left|\mathbf{n}\right|\left|\mathbf{n}'\right|\cos\theta=\left|\mathbf{n}'\right|^2,\,\cos\theta=\frac{\left|\mathbf{n}'\right|}{\left|\mathbf{n}\right|}=\frac{\left|\mathbf{a}'\times\mathbf{b}'\right|}{\left|\mathbf{a}\times\mathbf{b}\right|}\]
이로써 원하는 결과를 얻는다.
\[\left|\mathbf{a}\times\mathbf{b}\right|\cos\theta=\left|\mathbf{a}'\times\mathbf{b}'\right|\]

\subsection{21번: 평행육면체의 부피와 행렬식}
평행육면체의 밑면 넓이가 \(\left|\mathbf{x}\times\mathbf{y}\right|\)일 때, 밑면의 법선벡터는 \(\mathbf{n}=\mathbf{x}\times\mathbf{y}\)이다. 높이는 \(\mathbf{z}\)의 \(\mathbf{n}\)방향 성분이므로, 다음과 같다.
\[\left|\frac{\mathbf{n}\cdot\mathbf{z}}{\mathbf{n}\cdot\mathbf{n}}\mathbf{n}\right|=\left|\frac{\mathbf{n}\cdot\mathbf{z}}{\left|\mathbf{n}\right|}\right|=\left|\frac{(\mathbf{x}\times\mathbf{y})\cdot\mathbf{z}}{\left|\mathbf{x}\times\mathbf{y}\right|}\right|\]
따라서 부피를 구하면
\[V=\left|\mathbf{x}\times\mathbf{y}\right|\left|\frac{(\mathbf{x}\times\mathbf{y})\cdot\mathbf{z}}{\left|\mathbf{x}\times\mathbf{y}\right|}\right|=\left|(\mathbf{x}\times\mathbf{y})\cdot\mathbf{z}\right|=\left|\det(\mathbf{x},\mathbf{y},\mathbf{z})\right|\]

\subsection{23번: 선형사상과 평행육면체의 부피}
\(L_A\)를 보고 기억이 나지 않았을 수도 있는데, 이는 행렬 \(A\)에서 얻은 선형사상을 의미한다. 따라서 주어진 식은 다음과 같이 쓸 수 있다.
\[\text{Vol}(L_A(\mathbf{x}),L_A(\mathbf{y}),L_A(\mathbf{z}))=\text{Vol}(A\mathbf{x},A\mathbf{y},A\mathbf{z})=\left|\det(A\mathbf{x},A\mathbf{y},A\mathbf{z})\right|\]
여기에서, 이미 알고 있는\footnote{`기억이 잘 나지 않는' 이라고 읽으면 된다.} 행렬식의 성질 \(\det AB=(\det A)(\det B)\)를 이용하려면 다음이 성립함을 증명해야 한다.
\[A(\mathbf{x},\mathbf{y},\mathbf{z})=(A\mathbf{x},A\mathbf{y},A\mathbf{z})\]
첫 번째 행렬의 행과 두 번째 행렬의 열을 각각 내적함을 생각하면 쉽게 알 수 있다. 그래도 의심가면 한번 해 보자.\footnote{편의상 \(\mathbf{x}=(x_1,x_2,x_3)\)과 같이 문자를 잡았다.}
\[\begin{pmatrix}
a_{11} & a_{12} & a_{13} \\
a_{21} & a_{22} & a_{23} \\
a_{31} & a_{32} & a_{33}
\end{pmatrix}\begin{pmatrix}
x_1 & y_1 & z_1 \\
x_2 & y_2 & z_2 \\
x_3 & y_3 & z_3
\end{pmatrix}=(A\mathbf{x},A\mathbf{y},A\mathbf{z})\]
위 관계가 성립함을 알면, 간단하게 행렬식의 성질을 이용하여 원하는 결과를 얻는다.
\[\left|\det(A\mathbf{x},A\mathbf{y},A\mathbf{z})\right|=\left|\det A(\mathbf{x},\mathbf{y},\mathbf{z})\right|=\left|(\det A)(\det(\mathbf{x},\mathbf{y},\mathbf{z}))\right|=\left|\det A\right|\text{Vol}(\mathbf{x},\mathbf{y},\mathbf{z})\]
\end{document}
