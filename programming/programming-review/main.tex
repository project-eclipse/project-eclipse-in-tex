\documentclass{scrartcl}
\usepackage[margin=3cm]{geometry}
\usepackage{amsmath}
\usepackage{amssymb}
\usepackage{amsthm}
\usepackage{blindtext}
\usepackage{datetime}
\usepackage{fontspec}
\usepackage{graphicx}
\usepackage{hyperref}
\usepackage{kotex}
\usepackage{listings}
\usepackage[lighttt]{lmodern}
\usepackage{mathrsfs}
\usepackage{mathtools}
\usepackage{pgf,tikz,pgfplots}

\usepackage[headsepline]{scrlayer-scrpage}
\newcommand{\doctitle}{프로그래밍 코드 정리 Part. 0 - 파이썬 문법}
\clearpairofpagestyles
\ohead{\thepage}
\ihead{\doctitle}

\lstset{language=Python,
  numbers=none, frame=single, showspaces=false,
  showstringspaces=false, showtabs=false, breaklines=true, showlines=true,
  breakatwhitespace=true, basicstyle=\ttfamily, keywordstyle=\bfseries, basewidth=0.5em
}
\pgfplotsset{compat=1.15}
\usetikzlibrary{arrows}
\newtheorem{theorem}{Theorem}

\setmainhangulfont[ItalicFont={*},ItalicFeatures={FakeSlant=.22}]{Noto Serif CJK KR}
\setsansfont{NanumSquare}
\newfontfamily\sectionfont{NanumSquare Bold}
\addtokomafont{section}{\sectionfont\mdseries}
\addtokomafont{section}{\sectionfont\mdseries}
\addtokomafont{subsection}{\sectionfont\mdseries}
\addtokomafont{paragraph}{\sectionfont\mdseries}
\addtokomafont{title}{\sectionfont\mdseries}
\addtokomafont{author}{\sectionfont}
\title{\doctitle}
\author{Project Eclipse (19092 손량)}
\date{Last compiled on: \today, \currenttime}

\newcommand{\Lim}[1]{\lim_{#1\to\infty}}
\newcommand{\para}{\mathbin{\!/\mkern-5mu/\!}}
\newcommand{\Seg}[1]{\overline{#1}}
\newcommand{\Line}[1]{\overleftrightarrow{#1}}
\newcommand{\Ray}[1]{\overrightarrow{#1}}
\newcommand{\infsum}[1]{\sum^\infty_{#1}}
\newcommand{\mat}[3]{\begin{pmatrix}
#1_{11} & #1_{12} & \cdots & #1_{1#3} \\
#1_{21} & #1_{22} & \cdots & #1_{2#3} \\
\vdots & \vdots & \ddots & \vdots \\
#1_{#2 1} & #1_{#2 2} & \cdots & #1_{#2#3} \\
\end{pmatrix}}
\newcommand{\un}[1]{\ensuremath{\ \mathrm{#1}}}
\newcommand{\combi}[2]{{}_{#1}\mathrm{C}_{#2}}
\newcommand{\repcombi}[2]{{}_{#1}\mathrm{H}_{#2}}

\begin{document}
\maketitle

PPT 파일 3개와 266페이지 정도 되는 시험범위를 정리해 보자. 이 정리에서는 PPT에 나온 코드 그대로를 살펴보기보다는\footnote{정말 그런 것을 기대하는 거라면 그냥 PPT를 보자.} 코드에서 사용된 문법이나 방식이 왜 필요한지에 대해 좀 더 초점을 맞출 것이다.

Part 0는 프로그래밍 시간에 배운 파이썬 문법들에 대한 내용이다. 기본 파이썬 문법(조건문, 반복문, 함수, 클래스 등)은 이해한다고 가정하고 설명할 것이다.

\section{\texttt{\_\_name\_\_}과 \texttt{\_\_main\_\_}}
\texttt{\_\_main\_\_}은 현재 실행 중인 모듈의 이름이다. 이렇게만 설명하면 이해가 잘 가지 않을 수 있으니, 모듈부터 복습해 보자.

\subsection{파이썬 모듈}
파이썬 모듈은 함수나 변수의 선언 정보\footnote{물론 선언 정보 이외에 다른 코드도 넣을 수 있다.}를 담는 파일이다. 파이썬 모듈이 실행되는 방법에는 크게 두 가지가 있다.
\begin{itemize}
  \item \texttt{python} 명령어를 직접 입력해서 실행
  \item 다른 모듈에서 \texttt{import}
\end{itemize}
직접 실행하는 경우에는 \texttt{\_\_name\_\_}에는 \texttt{\_\_main\_\_}이 담기고, 다른 모듈에서 \texttt{import}하는 경우에는 모듈 이름이 담긴다. 예를 들어, \texttt{module.py}에 다음과 같은 코드가 있다고 하자.
\begin{lstlisting}
print(__name__)
\end{lstlisting}
이 코드를 \texttt{python module.py}와 같이 실행하면 \texttt{\_\_main\_\_}이 출력된다. 반면 \texttt{import module}과 같이 \texttt{import}하면 모듈의 이름인 \texttt{module}이 출력된다. 파이썬 프로그래밍에서 \texttt{\_\_name\_\_}은 주로 직접 실행될 때와 \texttt{import}될 때를 구분하기 위해 사용된다. 파이썬 코드를 직접 실행하는 경우는 그 모듈 자체를 직접 사용하는 사용자가 있는 상황이고, GUI 창을 띄운다던가, 로그를 기록하는 등 사용자의 입력을 받기 위한 처리가 필요하다. 반면 \texttt{import}되는 상황에서는 그러한 필요가 전혀 없고, 오히려 그 모듈을 사용하는 프로그램과 충돌을 일으킬 수 있기 때문에 사용자 입력을 받는 코드는 실행되지 않아야 한다. 이를 위해 자주 사용하는 방법으로 함수와 변수 선언이 끝난 뒤에 \texttt{\_\_name\_\_}을 검사하여 \texttt{\_\_main\_\_}인지 확인하는 것이 있다.
\begin{lstlisting}
def f(x, y):
    return x + y

if __name__ == '__main__':
    x = int(input())
    y = int(input())
    print(f(x, y))
\end{lstlisting}
모듈을 \texttt{import}하든 직접 실행하든 함수 \texttt{f}를 정의하는 코드는 실행된다. 모듈을 \texttt{import}하는 경우에는 보통 \texttt{f}의 정의만 필요하고, 모듈을 직접 실행할 때에는 사용자의 입력을 기대하기\footnote{사용자의 입력을 받지 않았다면 프로그램은 함수 정의 코드만 실행하고 그냥 종료된다.} 때문에 \texttt{\_\_name\_\_}이 \texttt{\_\_main\_\_}일 때만, 즉 직접 실행할 때만 숫자를 입력받아 더하기를 한다.

\section{스페셜 메소드}
파이썬이 객체 지향 언어라는 이야기는 몇 번 들어 보았을 것이다. 객체 지향이 무엇인지 모른다면, 파이썬의 데이터가 모두 클래스의 인스턴스라는 것이라고 이해하면 된다. 즉, 기본 타입인 \texttt{int}, \texttt{str}, \texttt{list}조차 그것의 클래스가 있어서\footnote{사실 이는 이해를 돕기 위해 한 말이고, 파이썬 기본 타입의 실제 구현은 다른 방식으로 되어 있다.}
\begin{lstlisting}
class int:
    def __init__(self, x, *args, **kwargs):
        # some implementation...

    # more implementation...

class str:
    def __init__(self, x, *args, **kwargs):
        # yet another implementation...

    # more implementation...
\end{lstlisting}
이들을 사용할 때 사실 클래스로부터 인스턴스를 찍어 낸다는 것이다.
\begin{lstlisting}
x = 42  # same as x = int(42)
y = 'Hell world!'  # same as y = str('Hell world!')
\end{lstlisting}
또한 사칙연산이나 원소를 가져오는 \texttt{[]}와 같이 객체를 가지고 하는 연산들은 모두 메소드의 형태로 구현되어 있다.
\begin{lstlisting}
print(x + 10)  # same as print(x.__add__(10))
print(y[-1])  # same as print(y.__getitem__(-1))
\end{lstlisting}
\texttt{\_\_add\_\_}와 같은 메소드를 특수 메소드라고 부르고, 직접 실행할 수도 있지만 연산자와 같은 파이썬의 언어적 기능을 통해서도 실행할 수 있다. 이러한 메소드가 필요한 이유는 사용자가 직접 정의한 타입도 파이썬 언어에 내장된 기능을 사용할 수 있도록 하기 위해서이다. 예를 들어, 파이썬 \texttt{list}와는 비슷하지만 약간 다른 기능을 가진 \texttt{array}를 만들었다고 하자. 약간 다른 기능을 가졌다고 해도, \texttt{list}가 할 수 있는 대부분의 것을 \texttt{array}도 할 수 있어야 헷갈리지 않고 코딩이 편할 것이다. 두 개의 \texttt{array}를 합칠 때 다음과 같이 코딩해야 편하지
\begin{lstlisting}
arr1 = array([1, 2, 3])
arr2 = array([4, 5, 6])
arr3 = arr1 + arr2
\end{lstlisting}
이렇게 \texttt{array}를 합치기 위해 다른 방법을 사용한다면 매우 불편할\footnote{그냥 \texttt{list}와 \texttt{array}를 상황에 맞게 구분하면 되니까 큰 문제가 아닐 것이라고 생각할 수 있겠지만, 문제는 파이썬이 동적 타이핑 언어이기 때문에 `상황에 맞게 구분'하는 것이 꽤나 어렵다는 것이다. 코드를 짜는 시점에서 예상하고 있던 타입이 코드를 실행하는 시점에서는 그와 다를 가능성이 매우 높다. 그러한 경우에 아무런 처리를 하지 않았다면 프로그램은 예외가 발생하고 비정상 종료될 것이다. 이러한 형태의 버그는 처리되는 데이터의 타입을 계속 추적해야 하기 때문에 찾기가 쉽지 않다.} 것이다.
\begin{lstlisting}
arr1 = array([1, 2, 3])
arr2 = array([4, 5, 6])
arr3 = arr1.concat(arr2)
\end{lstlisting}

\section{Iterable 객체}
파이썬의 \texttt{list}와 \texttt{tuple}, \texttt{str} 등의 객체는 모두 \texttt{for}문을 통해 순회할 수 있다.
\begin{lstlisting}
ls = [1, 2, 3, 4]
for elem in ls:
    print(elem)
\end{lstlisting}
클래스를 통해 직접 정의한 타입도 이들처럼 \texttt{for}문을 사용할 수 있게 하기 위해서는 iterable 객체가 되어야 한다. iterable 객체가 되기 위해서는 스페셜 메소드 \texttt{\_\_iter\_\_}가 있으면 된다. 이 메소드는 파이썬이 객체의 원소를 하나씩 순서대로 참조할 수 있게 하는 iterator를 반환하면 된다. iterator는 다음 원소를 가져오는 \texttt{\_\_next\_\_} 스페셜 메소드를 구현하는 객체이다. 이 \texttt{\_\_next\_\_} 메소드는 반복하는데 사용될 원소를 하나씩 반환하고, 더 이상 줄 원소가 없을 때에는 \texttt{StopIteration} 예외를 발생시키면 된다. 두 메소드는 스페셜 메소드이기 때문에, 직접 메소드명을 입력해서 실행할 수도 있지만 보통 다음과 같이 실행한다.
\begin{lstlisting}
i = iter(ls)  # same as i = ls.__iter__()
x = next(i)  # same as x = i.__next__()
\end{lstlisting}
한 가지 예로, 원소를 작은 것부터 큰 것까지 순서대로 저장하는 리스트 \texttt{SortedList}를 만든다고 하자.\footnote{\texttt{bisect}는 리스트에 원소를 추가하는 작업을 효율적으로 하기 위해 사용했다. 정보/프로그래밍 시간에 배운 적이 없으니까 시험에도 나오지 않고, 이해가 안 되도 크게 상관없는 부분이니 넘어가자.}\footnote{\texttt{\_\_len\_\_}은 \texttt{len(ls)}와 같이 \texttt{len} 함수를 사용할 때 실행되는 스페셜 메소드이다.}
\begin{lstlisting}
import bisect

class SortedList:
    def __init__(self, ls=[]):
        self.ls = ls
        self.ls.sort()

    def add(self, val):
        idx = bisect.bisect_right(self.ls, val)
        self.ls.insert(idx, val)

    def __getitem__(self, idx):
        return self.ls[idx]
    
    def __len__(self):
        return len(self.ls)
\end{lstlisting}
이것이 iterable하기 위해서는 이터레이터를 반환하는 \texttt{\_\_iter\_\_} 메소드가 필요하다. 이터레이터 \texttt{SortedListIterator}까지 직접 구현해 보면 다음과 같다.
\begin{lstlisting}
import bisect

class SortedList:
    def __init__(self, ls=[]):
        self.ls = ls
        self.ls.sort()

    def add(self, val):
        idx = bisect.bisect_right(self.ls, val)
        self.ls.insert(idx, val)

    def __getitem__(self, idx):
        return self.ls[idx]
    
    def __len__(self):
        return len(self.ls)

    def __iter__(self):
        return SortedListIterator(self)

class SortedListIterator:
    def __init__(self, ls: SortedList):
        self.ls = ls
        self.idx = 0

    def __next__(self):
        if self.idx >= len(self.ls):
            raise StopIteration
        self.idx += 1
        return self.ls[self.idx - 1]
\end{lstlisting}
\texttt{\_\_iter\_\_}에서 \texttt{SortedListIterator} 객체를 만들어 반환해 주는 것을 볼 수 있다. 이름에 나온 것처럼 이것이 iterator이고, \texttt{\_\_next\_\_} 메소드에서 차례차례 하나씩 원소를 반환하는 것을 볼 수 있다. 이렇게 코드를 작성하면 다음과 같이 \texttt{for} 루프로 원소 하나씩 순회하는 것이 가능하다.
\begin{lstlisting}
ls = SortedList()
ls.add(2)
ls.add(3)
ls.add(5)
ls.add(7)
ls.add(11)
for elem in ls:
    print(elem)
\end{lstlisting}
즉, 알고 보면 다음과 같은 \texttt{for} 루프는
\begin{lstlisting}
ls = [1, 1, 2, 3, 5, 8, 13, 21]
for elem in ls:
    print(ls)
\end{lstlisting}
이 코드와 똑같다는 것이다.
\begin{lstlisting}
ls = [1, 1, 2, 3, 5, 8, 13, 21]
i = iter(ls)
while True:
    try:
        elem = next(i)
        print(elem)
    except StopIteration:
        break
\end{lstlisting}

\section{타입 어노테이션}
파이썬은 동적 타이핑 언어이다. 다르게 말하면, 변수에 어떤 타입이든 넣을 수 있다는 뜻이다. 이 특성은 파이썬에게 매우 큰 장점이자 단점이다. 잘 사용하면 자바나 C언어에서 하는 것과 같이 타입을 일일이 선언할\footnote{최신 버전의 자바는 타입을 자동으로 추론하는 기능이 있기는 하다.} 필요가 없지만, 잘못 사용하면 예측하기 힘든 타입 관련 예외가 발생할 수 있다. 이러한 문제점을 해소하기 위해 파이썬 3.5에서 3.6 버전에서는 타입 어노테이션이라는 기능이 등장했다. 이 타입 어노테이션에서는 함수의 인자와 반환값, 변수의 타입을 지정할 수 있게 해 준다. 예를 들어 \texttt{x}로 \texttt{int}를 받고 \texttt{y}로 \texttt{str}을 받아 \texttt{bool}을 반환하는 함수에 대해 다음과 같은 선언을
\begin{lstlisting}
def f(x, y):
\end{lstlisting}
타입 어노테이션을 넣어 다음과 같이 바꿀 수 있다.
\begin{lstlisting}
def f(x: int, y: str) -> bool:
\end{lstlisting}
타입 어노테이션은 파이썬 코드의 실행에 전혀 영향을 끼치지 않는다.\footnote{심지어 \texttt{int}라고 표시해 놓은 변수에 \texttt{str} 타입 데이터를 넣어도 상관 없다.} 타입 어노테이션에 적은 타입 정보는 주로 파이썬 개발 도구가 참조하는데, 이 정보를 기반으로 자동 완성, 코드 검사 도구가 더 정확한 결과를 줄 수 있도록 한다. 예를 들어\footnote{여기서부터 시험과 관련 없는 내용이니 스킵해도 된다.} 다음과 같은 함수가 있다고 해 보자. 띄어쓰기로 구분된 숫자들이 저장된 문자열을 숫자 리스트로 바꿔 주는 함수이다.
\begin{lstlisting}
def str_to_intlist(s):
    result = []
    for elem in s.split():
        result.append(int(elem))
    return result
\end{lstlisting}
이 함수를 PyCharm이나 Visual Studio Code같은 자동완성을 지원하는 편집기에 직접 쳐 보면, \texttt{s.split}을 칠 때 자동완성이 되지 않을 것이다. 이는 자동완성 엔진에서 \texttt{s}의 타입이 무엇인지 모르기 때문이다. 반면 다음과 같이 코드를 작성하면\footnote{\texttt{typing} 모듈은 \texttt{int}를 담는 리스트라는 것을 알려 주기 위해 사용했고, 몰라도 상관없다.}
\begin{lstlisting}
from typing import List
def str_to_intlist(s: str) -> List[int]:
    result = []
    for elem in s.split():
        result.append(int(elem))
    return result
\end{lstlisting}
편집기가 자동완성을 해 주는 것을 볼 수 있다. 이러한 이유로 편집기의 타입 정보는 아무런 문자열을 써놓기보다는 개발 도구들이 해석할 수 있도록 파이썬 기본 타입을 입력하거나, \texttt{typing} 모듈을 활용하는 것이 가장 좋을 것이다.

\end{document}
