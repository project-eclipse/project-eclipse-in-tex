\documentclass{scrartcl}
\usepackage[margin=3cm]{geometry}
\usepackage{amsmath}
\usepackage{amssymb}
\usepackage{amsthm}
\usepackage{blindtext}
\usepackage{datetime}
\usepackage{fancyvrb}
\usepackage{fontspec}
\usepackage{graphicx}
\usepackage{hyperref}
\usepackage{kotex}
\usepackage{mathrsfs}
\usepackage{mathtools}
\usepackage{pifont}
\usepackage{pgf,tikz,pgfplots}

\usepackage[headsepline]{scrlayer-scrpage}
\newcommand{\doctitle}{수학세미나 2쿼터 HW E 풀이}
\clearpairofpagestyles
\ohead{\thepage}
\ihead{\doctitle}

\hypersetup{
    colorlinks=true,
    linkcolor=black,
    urlcolor=blue,
}
\pgfplotsset{compat=1.15}
\usetikzlibrary{arrows}
\newtheorem{theorem}{Theorem}

\setmainhangulfont[ItalicFont={*},ItalicFeatures={FakeSlant=.22}]{Noto Serif CJK KR}
\setsansfont{NanumSquare}
\newfontfamily\sectionfont{NanumSquare Bold}
\addtokomafont{section}{\sectionfont\mdseries}
\addtokomafont{subsection}{\sectionfont\mdseries}
\addtokomafont{subsubsection}{\sectionfont\mdseries}
\addtokomafont{paragraph}{\sectionfont\mdseries}
\addtokomafont{title}{\sectionfont\mdseries}
\addtokomafont{author}{\sectionfont}
\title{\doctitle}
\author{Project Eclipse (19092 손량)}
\date{Last compiled on: \today, \currenttime}

\newcommand{\Lim}[1]{\lim_{#1\to\infty}}
\newcommand{\para}{\mathbin{\!/\mkern-5mu/\!}}
\newcommand{\Seg}[1]{\overline{#1}}
\newcommand{\Line}[1]{\overleftrightarrow{#1}}
\newcommand{\Ray}[1]{\overrightarrow{#1}}
\newcommand{\infsum}[1]{\sum^\infty_{#1}}
\newcommand{\mat}[3]{\begin{pmatrix}
#1_{11} & #1_{12} & \cdots & #1_{1#3} \\
#1_{21} & #1_{22} & \cdots & #1_{2#3} \\
\vdots & \vdots & \ddots & \vdots \\
#1_{#2 1} & #1_{#2 2} & \cdots & #1_{#2#3} \\
\end{pmatrix}}
\newcommand{\un}[1]{\ensuremath{\ \mathrm{#1}}}
\newcommand{\combi}[2]{{}_{#1}\mathrm{C}_{#2}}
\newcommand{\repcombi}[2]{{}_{#1}\mathrm{H}_{#2}}
\newcommand{\condprob}[2]{P\left({#1}\;\middle|\;{#2}\right)}
\newcommand{\cmark}{\ding{51}}
\newcommand{\xmark}{\ding{55}}

\begin{document}
\maketitle

\section{1명씩 장소 2개에서 감염자가 발생할 때 확률 구하기}
우선 마스크를 끼는 사건도 있고, 감염되는 사건도 있어 복잡하기 때문에 조건부확률을 이용해 감염될 확률을 구하자. 문제에서 주어진 조건은 다음과 같이 나타낼 수 있다.
\begin{align*}
  \condprob{\text{감염}}{\text{마스크}}=\frac{1}{10},\,\condprob{\text{감염}}{\text{마스크}^C}=\frac{1}{2},\,P\left(\text{마스크}\right)=\frac{1}{2}
\end{align*}
감염될 확률을 구하면 다음과 같다.
\begin{align*}
  P\left( \text{감염} \right)&=P\left( \text{감염}\cap\text{마스크} \right)+P\left( \text{감염}\cap\text{마스크}^C \right) \\
  &=P\left(\text{마스크}\right)\condprob{\text{감염}}{\text{마스크}}+P\left(\text{마스크}^C\right)\condprob{\text{감염}}{\text{마스크}^C}=\frac{3}{10}
\end{align*}
A에서 감염자가 1명 발생할 확률은 다음과 같다. 장소 A에서는 모든 사람이 마스크를 끼지 않는다는 것에 유의하자.
\begin{align*}
  P(\mathrm{A})=\combi{3}{1}\left( \frac{1}{2} \right)^1\left( \frac{1}{2} \right)^2=\frac{3}{8}
\end{align*}
비슷하게 계산하면 B, C에서 감염자가 1명 발생할 확률은 다음과 같다.
\begin{align*}
  P(\mathrm{B})=\combi{2}{1}\left( \frac{3}{10} \right)^1\left( \frac{7}{10} \right)^1=\frac{21}{50}, P(\mathrm{C})=\frac{3}{10}
\end{align*}
A, B, C에서 감염자가 발생하지 않을 확률은 각각 다음과 같다. 단순히 1에서 먼저 구한 확률을 빼서 구하지 않도록 주의하자.
\begin{align*}
  P(\mathrm{A}')=\left( \frac{1}{2} \right)^3=\frac{1}{8}, P(\mathrm{B}')=\left( \frac{7}{10} \right)^2=\frac{49}{100}, P(\mathrm{C}')=1-(\mathrm{C})=\frac{7}{10}
\end{align*}
구하는 확률은 다음과 같다.
\begin{align*}
  P(\mathrm{A}')P(\mathrm{B})P(\mathrm{C})+P(\mathrm{A})P(\mathrm{B}')P(\mathrm{C})+P(\mathrm{A})P(\mathrm{B})P(\mathrm{C}')=\frac{1449}{8000}
\end{align*}
따라서 \(\alpha=23\)이다.

\section{경우의 수 구하기}
문제를 풀기 전, 표본 공간에 대한 고려를 할 필요가 있다. 경우의 수나 세면 됐지 쓸데없는 생각은 왜 하나 할 수 있겠지만, 다음 질문을 생각해 보자.
\begin{quote}
  A와 B에서 감염자가 한 명씩 발생한 경우는 몇 가지로 세야 할까?
\end{quote}
언뜻 생각하면 1가지라고 생각할 수 있지만\footnote{나만 그렇게 생각했나...? 혹시 과소평가당한다는 기분이 든다면 너그럽게 넘어가자.} 여기서 A에 사람이 3명 있고, B에는 사람이 2명 있을뿐더러 C에 있는 사람이 마스크를 착용하는지도 고려해야 한다. 그렇게 간단한 문제가\footnote{애초에 이런 식으로 풀린다면 수학세미나에서 다루지도 않을 것이다.} 아닌 것이다. 이 문제에서 6명의 사람\footnote{이 문제를 풀면서 처음 알았는데 보통 사람은 서로 다른 것으로 고려한다고 한다.}이 마스크를 낀다/안낀다의 경우가 있고, 1번 사람에서 6번 사람까지 감염자가 총 2명 나오므로 표본 공간은\footnote{컴퓨터에 배경이 있다면 \Verb|bool| 타입 원소가 6개, \Verb|int| 타입 원소가 2개인 튜플로 생각할 수도 있을 것이다.} 다음과 같이 나타낼 수 있다.\footnote{물론 표본 공간을 나타내는 방법은 이것 말고도 여러 가지가 있을 것이다.} \cmark는 마스크를 쓴 것을, \xmark는 마스크를 쓰지 않은 것을 나타낸다.
\begin{align*}
  S=\{(\text{\xmark}, \text{\xmark}, \text{\xmark}, \text{\cmark}, \text{\xmark}, \text{\cmark}, 1, 5),\dots\}
\end{align*}
또한, A, B, C에 있는 사람들이 마스크를 낀 상태에 대한 고려도 해야 한다. A, B, C의 사람들이 마스크를 끼는 가능성에 대한 집합을 각각 \(S_A, S_B, S_C\)라고 하면 다음과 같다.
\begin{align*}
  S_A=\{(\text{\xmark}, \text{\xmark}, \text{\xmark})\}, S_B=\{(\text{\cmark}, \text{\xmark}), (\text{\xmark}, \text{\cmark}), (\text{\cmark}, \text{\cmark})\}, S_C=\{(\text{\cmark}), (\text{\xmark})\}
\end{align*}
무엇을 구할지 알았으니, 케이스 분류를 해 보자. 마스크 착용자 중 감염자는 1명 이하이므로 다음과 같이 나눠진다.

\subsection{B에서 마스크를 착용한 사람이 감염된 경우}
우선 이 경우 \(S_B\)에 있는 경우들 중 3가지 모두 가능하고, \((\text{\cmark}, \text{\cmark})\)의 경우에는 감염자가 나오는 경우가 2가지이다. A에서 감염자가 나온 경우 감염자는 3명 중 한 명이고 C에서 마스크를 착용하는 가짓수가 2가지이다. C에서 감염자가 나온 경우 감염자는 조건에 의해 무조건 마스크를 착용하지 않아야 하므로 다음과 같이 계산된다.
\begin{align*}
  4\times(3\times 2+1)=28
\end{align*}

\subsection{C에서 마스크를 착용한 사람이 감염된 경우}
C의 감염자는 무조건 마스크를 착용하므로 \(S_C\)에서 한 가지만 가능하다. A에서 감염자가 나오면 3명 중 한명이 감염되어야 하고, B에서 마스크 착용 상태가 \(n(S_B)=3\)가지 가능하다. 한편 B에서 감염자가 나오면 B의 마스크 착용 상태는 \((\text{\cmark}, \text{\cmark})\)를 제외하고 \(n(S_B)-1=2\)가지 가능하므로 다음과 같이 계산하면 된다.
\begin{align*}
  1\times(3\times 3+2)=11
\end{align*}

\subsection{마스크를 착용한 감염자가 없는 경우}
A에서 감염자가 나온다면 A에서는 3명 중 한명이 감염될 수 있다. B에서 감염자가 나온다면 마스크 착용 상태 중 \((\text{\cmark}, \text{\cmark})\)를 제외하고 \(n(S_B)-1=2\)가지가 가능하다. C에서 감염자가 나온다면 마스크를 착용하지 않는 경우 1가지만 가능하다. 반면, A에서 감염자가 나오지 않는 경우는 1가지이고, B에서 감염자가 나오지 않는 경우는 \(n(S_B)=3\)가지, C에서 감염자가 나오지 않는 경우는 2가지이므로 다음과 같이 계산하면 된다.
\begin{align*}
  3\times 2\times 2+3\times 3\times 1+1\times 2\times 1=23
\end{align*}
따라서 구하는 경우의 수는 \(28+11+23=62\)이다.

\end{document}
