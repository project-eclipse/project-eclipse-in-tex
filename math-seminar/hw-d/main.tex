\documentclass{scrartcl}
\usepackage[left=2cm, right=2cm, top=2cm, bottom=2cm]{geometry}
\usepackage{amsmath}
\usepackage{amssymb}
\usepackage{amsthm}
\usepackage{blindtext}
\usepackage{datetime}
\usepackage{fontspec}
\usepackage{float}
\usepackage{graphicx}
\usepackage{hyperref}
\usepackage{mathrsfs}
\usepackage{mathtools}
\usepackage{pgf,tikz,pgfplots}

\usepackage[headsepline]{scrlayer-scrpage}
\newcommand{\doctitle}{수학세미나 HW D 풀이}
\clearpairofpagestyles
\ohead{\thepage}
\ihead{\doctitle}

\hypersetup{
    colorlinks=true,
    linkcolor=black,
    urlcolor=blue,
}
\pgfplotsset{compat=1.15}
\usetikzlibrary{arrows}
\newtheorem{theorem}{Theorem}

\setmainfont{Noto Serif CJK KR}
\setsansfont{NanumSquare}
\newfontfamily\sectionfont{NanumSquare Bold}
\addtokomafont{section}{\sectionfont\mdseries}
\addtokomafont{subsection}{\sectionfont\mdseries}
\addtokomafont{subsubsection}{\sectionfont\mdseries}
\addtokomafont{paragraph}{\sectionfont\mdseries}
\addtokomafont{title}{\sectionfont\mdseries}
\addtokomafont{author}{\sectionfont}
\title{\doctitle}
\author{Project Eclipse (손량)}
\date{Last compiled on: \today, \currenttime}

\newcommand{\Lim}[1]{\lim_{#1\to\infty}}
\newcommand{\para}{\mathbin{\!/\mkern-5mu/\!}}
\newcommand{\Seg}[1]{\overline{#1}}
\newcommand{\Line}[1]{\overleftrightarrow{#1}}
\newcommand{\Ray}[1]{\overrightarrow{#1}}
\newcommand{\infsum}[1]{\sum^\infty_{#1}}
\newcommand{\mat}[3]{\begin{pmatrix}
#1_{11} & #1_{12} & \cdots & #1_{1#3} \\
#1_{21} & #1_{22} & \cdots & #1_{2#3} \\
\vdots & \vdots & \ddots & \vdots \\
#1_{#2 1} & #1_{#2 2} & \cdots & #1_{#2#3} \\
\end{pmatrix}}
\newcommand{\un}[1]{\ensuremath{\ \mathrm{#1}}}

\begin{document}
\maketitle

\section{위치벡터 구하기}
\begin{figure}[H]
\centering
\documentclass[10pt]{article}
\usepackage{pgfplots}
\pgfplotsset{compat=1.15}
\usepackage{mathrsfs}
\usetikzlibrary{arrows}
\pagestyle{empty}
\begin{document}
\definecolor{xdxdff}{rgb}{0.49019607843137253,0.49019607843137253,1}
\definecolor{uuuuuu}{rgb}{0.26666666666666666,0.26666666666666666,0.26666666666666666}
\definecolor{qqwuqq}{rgb}{0,0.39215686274509803,0}
\begin{tikzpicture}[line cap=round,line join=round,>=triangle 45,x=1cm,y=1cm]
\begin{axis}[
x=1cm,y=1cm,
axis lines=middle,
ymajorgrids=true,
xmajorgrids=true,
xmin=-5.040000000000002,
xmax=5.040000000000002,
ymin=-6.94,
ymax=6.94,
xtick={-5,-4,...,5},
ytick={-6,-5,...,6},]
\clip(-5.04,-6.94) rectangle (5.04,6.94);
\draw[line width=2pt,color=qqwuqq,smooth,samples=100,domain=-5.040000000000002:5.040000000000002] plot(\x,{cos(((\x))*180/pi)});
\draw [line width=2pt] (0.9,0.6216099682706644)-- (0.9,0);
\begin{scriptsize}
\draw[color=qqwuqq] (-4.86,0.19) node {$f$};
\draw [fill=uuuuuu] (-1.5707963267924883,2.408357017356847E-12) circle (2pt);
\draw[color=uuuuuu] (-1.42,0.39) node {$A$};
\draw [fill=uuuuuu] (1.5707963267948875,0) circle (2pt);
\draw[color=uuuuuu] (1.74,0.39) node {$B$};
\draw [fill=xdxdff] (0.9,0.6216099682706644) circle (2.5pt);
\draw[color=xdxdff] (1.06,1.05) node {$P$};
\draw [fill=uuuuuu] (0.9,0) circle (2pt);
\draw[color=uuuuuu] (1.14,0.39) node {$P'$};
\draw[color=black] (0.64,0.55) node {$h$};
\end{scriptsize}
\end{axis}
\end{tikzpicture}
\end{document}
\end{figure}
문제에서 주어진 조건만으로 해결 가능하다.
\begin{align*}
\Ray{BP}&=\Ray{BO}+\Ray{OP}=-\vec{b}+\vec{p} \\
\Ray{PH}&=\Ray{PO}+\Ray{OH}=-\vec{p}+(\vec{a}\cdot\vec{p})\vec{a}
\end{align*}

\section{항등식 증명}
포물선의 정의에 의해, \(\Seg{BP}=\Seg{PH}\)이므로 다음이 성립한다.
\begin{align*}
|\Ray{BP}|^2=&(-\vec{b}+\vec{p})^2=|\vec{b}|^2-2\vec{b}\cdot\vec{p}+|\vec{p}|^2=1-2\vec{b}\cdot\vec{p}+|\vec{p}|^2 \\
=&|\Ray{PB}|^2=|\vec{p}|^2-2(\vec{a}\cdot\vec{p})^2+(\vec{a}\cdot\vec{p})^2|\vec{a}|^2=|\vec{p}|^2-2(\vec{a}\cdot\vec{p})^2+(\vec{a}\cdot\vec{p})^2
\end{align*}
식을 변형하면 원하는 결과를 얻는다.
\begin{align}\label{identity}(\vec{p}\cdot\vec{a})^2-2(\vec{p}\cdot\vec{b})+1=0\end{align}

\section{\(\theta\) 구하기}
이 문제를 해결하는 방법에는 여러 가지가 있지만, 여기에서는 벡터만을 사용하는 방법을 다루겠다.

\(\Seg{OP}\)가 \(\angle AOB\)를 이등분한다는 조건에 주목하자. \(\Seg{OA}=\Seg{OB}=1\)이기 때문에, \(\Ray{OP}\)는 \(\Ray{OA}\)와 \(\Ray{OB}\)의 합의 실수배로 나타낼 수 있다. 따라서, \(\vec{p}\)를 다음과 같이 나타낼 수 있다.
\[\vec{p}=t(\vec{a}+\vec{b})\]
문제 조건에 의해 \(|\vec{p}|=2\)이므로, 다음이 성립한다.
\[|\vec{p}|^2=t^2(|\vec{a}|^2+2\vec{a}\cdot\vec{b}+|\vec{b}|^2)=t^2(2+2\vec{a}\cdot\vec{b})=4\]
식을 정리하면 다음과 같다.
\begin{align}\label{length_relation}t^2(\vec{a}\cdot\vec{b}+1)=2\end{align}
한편, (\ref{identity})에 \(\vec{p}\)의 정의식을 대입하면 다음과 같은 식을 얻는다.
\begin{align*}
\left[t(\vec{a}+\vec{b})\cdot\vec{a}\right]^2-2\left[t(\vec{a}+\vec{b})\cdot\vec{b}\right]+1&=\left[t(1+\vec{a}\cdot\vec{b})\right]^2-2\left[t(\vec{a}\cdot\vec{b}+1)\right]+1 \\
&=\left[t(1+\vec{a}\cdot\vec{b})-1\right]^2=0
\end{align*}
따라서, 다음과 같은 관계식을 얻을 수 있고
\begin{align}\label{identity_relation}t(1+\vec{a}\cdot\vec{b})=1\end{align}
(\ref{length_relation}), (\ref{identity_relation})의 관계를 통해 \(t=2\)임을 알 수 있다.
여기서 \(\Seg{OP}=2\)임을 이용하면
\[|\vec{p}|^2=t^2(|\vec{a}|^2+2\vec{a}\cdot\vec{b}+|\vec{b}|^2)=4(2+2\vec{a}\cdot\vec{b})=4\]
\(\vec{a}\cdot\vec{b}=|\vec{a}||\vec{b}|\cos\theta=-1\)이고, \(\cos\theta=-\dfrac{1}{2}\)이므로 \(\theta=\dfrac{2\pi}{3}\)이다.

\end{document}
