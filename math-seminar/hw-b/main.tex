\documentclass{scrartcl}
\usepackage[left=2cm, right=2cm, top=2cm, bottom=2cm]{geometry}
\usepackage{amsmath}
\usepackage{amssymb}
\usepackage{amsthm}
\usepackage{blindtext}
\usepackage{datetime}
\usepackage{float}
\usepackage{fontspec}
\usepackage{graphicx}
\usepackage{hyperref}
\usepackage{mathrsfs}
\usepackage{mathtools}
\usepackage{pgf,tikz,pgfplots}

\usepackage[headsepline]{scrlayer-scrpage}
\newcommand{\doctitle}{수학세미나 HW B 풀이}
\clearpairofpagestyles
\ohead{\thepage}
\ihead{\doctitle}

\hypersetup{
    colorlinks=true,
    linkcolor=black,
    urlcolor=blue,
}
\pgfplotsset{compat=1.15}
\usetikzlibrary{arrows}
\newtheorem{theorem}{Theorem}

\setmainfont{Noto Serif CJK KR}
\setsansfont{NanumSquare}
\newfontfamily\sectionfont{NanumSquare Bold}
\addtokomafont{section}{\sectionfont\mdseries}
\addtokomafont{subsection}{\sectionfont\mdseries}
\addtokomafont{subsubsection}{\sectionfont\mdseries}
\addtokomafont{paragraph}{\sectionfont\mdseries}
\addtokomafont{title}{\sectionfont\mdseries}
\addtokomafont{author}{\sectionfont}
\title{\doctitle}
\author{Project Eclipse (손량)}
\date{Last compiled on: \today, \currenttime}

\newcommand{\Lim}[1]{\lim_{#1\to\infty}}
\newcommand{\para}{\mathbin{\!/\mkern-5mu/\!}}
\newcommand{\Seg}[1]{\overline{#1}}
\newcommand{\Line}[1]{\overleftrightarrow{#1}}
\newcommand{\Ray}[1]{\overrightarrow{#1}}
\newcommand{\infsum}[1]{\sum^\infty_{#1}}
\newcommand{\mat}[3]{\begin{pmatrix}
#1_{11} & #1_{12} & \cdots & #1_{1#3} \\
#1_{21} & #1_{22} & \cdots & #1_{2#3} \\
\vdots & \vdots & \ddots & \vdots \\
#1_{#2 1} & #1_{#2 2} & \cdots & #1_{#2#3} \\
\end{pmatrix}}
\newcommand{\un}[1]{\ensuremath{\ \mathrm{#1}}}

\begin{document}
\maketitle

\section{삼각형의 분할비}
우선 삼각형을 다음과 같이 잡자.
\begin{figure}[H]
\centering
\definecolor{uuuuuu}{rgb}{0.26666666666666666,0.26666666666666666,0.26666666666666666}
\begin{tikzpicture}[line cap=round,line join=round,>=triangle 45,x=0.75cm,y=0.75cm]
\clip(-0.5,-0.5) rectangle (11.5,7.6);
\draw [line width=1pt] (8,7)-- (0,0);
\draw [line width=1pt] (0,0)-- (11,0);
\draw [line width=1pt] (11,0)-- (8,7);
\draw [line width=1pt] (8,7)-- (7,0);
\draw [line width=1pt] (11,0)-- (5.001061946902656,4.375929203539823);
\draw [line width=1pt] (0,0)-- (9.536160806796989,3.415624784140361);
\begin{scriptsize}
\draw [fill=uuuuuu] (0,0) circle (2pt);
\draw[color=uuuuuu] (0,0.39) node {$A$};
\draw [fill=black] (11,0) circle (2pt);
\draw[color=black] (11.16,0.43) node {$B$};
\draw [fill=black] (8,7) circle (2pt);
\draw[color=black] (8.16,7.43) node {$C$};
\draw [fill=black] (7,0) circle (2pt);
\draw[color=black] (7.3,0.3) node {$R$};
\draw [fill=black] (5.001061946902656,4.375929203539823) circle (2pt);
\draw[color=black] (5,4.81) node {$Q$};
\draw [fill=uuuuuu] (7.377491583391226,2.6424410837385777) circle (2pt);
\draw[color=uuuuuu] (7.65,3) node {$O$};
\draw [fill=uuuuuu] (9.536160806796989,3.415624784140361) circle (2pt);
\draw[color=uuuuuu] (9.7,3.81) node {$P$};
\end{scriptsize}
\end{tikzpicture}

\end{figure}
주어진 식을 변형하면 다음과 같은 식이 나온다.
\begin{align}\label{prop}\frac{l\Ray{OA}+m\Ray{OB}}{l+m}+\frac{n\Ray{OC}}{l+m}=0\end{align}
여기서, \(\Ray{OC}\)는 \(\dfrac{l\Ray{OA}+m\Ray{OB}}{l+m}\)의 상수배이므로, \(R\)은 \(\Seg{AB}\)의 \(m:l\) 내분점이다. 비슷한 논리로 다음을 보일 수 있다.
\begin{itemize}
 \item \(P\)는 \(\Seg{BC}\)의 \(n:m\) 내분점
 \item \(Q\)는 \(\Seg{CA}\)의 \(l:n\) 내분점
\end{itemize}
한편, (\ref{prop})의 식에 의하면 \(\Seg{CO}:\Seg{OR}=(l+m):n\)이므로, 삼각형의 넓이 비를 구할 수 있다.
\[\triangle CAR:\triangle CBR=\frac{l+m+n}{l+m}\triangle OCA:\frac{l+m+n}{l+m}\triangle OBC=\triangle OCA:\triangle OBC=m:l\]
또한, 비슷한 방식으로 \(\Seg{AO}:\Seg{OP}=(m+n):l\)이므로 다음을 얻는다.
\[\triangle OCA:\triangle OAB=m:n\]
따라서 원하는 비율을 얻는다.
\begin{align}\label{area}\triangle OBC:\triangle OCA:\triangle OAB=l:m:n\end{align}

\section{직접 숫자 넣어서 계산하기}
우선 \(\Ray{OC}\)를 구하자.
\[\Ray{OC}=-4\Ray{OA}-\frac{3}{2}\Ray{OB}\]
이를 \(\Ray{OA}\)와 내적하자.
\[\Ray{OA}\cdot\Ray{OC}=\Ray{OA}\cdot\left( -4\Ray{OA}-\frac{3}{2}\Ray{OB} \right)=-4\left|\Ray{OA}\right|^2-\frac{3}{2}\Ray{OA}\cdot\Ray{OB}=-12-\frac{3}{2}\Ray{OA}\cdot\Ray{OB}=-3\]
따라서 \(\Ray{OA}\cdot\Ray{OB}=-6\)이고, 한편 \(\Ray{OA}\)와 \(\Ray{OB}\)의 사잇각 \(\theta\)에 대해 \(\Ray{OA}\cdot\Ray{OB}=\Seg{OA}\cdot\Seg{OB}\cdot\cos\theta\)이므로, \(\cos\theta=-\sqrt{3}/2\)이다. \(\sin\theta=1/2\)이므로, 넓이를 다음과 같이 구할 수 있다.
\[\triangle OAB=\frac{1}{2}\Seg{OA}\cdot\Seg{OB}\cdot\sin\theta=\sqrt{3}\]
한편, (\ref{area})의 식을 사용하면 \(\triangle OBC:\triangle OCA:\triangle OAB=8:3:2\)이고, 따라서 구하는 넓이는 \(\dfrac{13\sqrt{3}}{2}\)이다.

\section{넓이비와 벡터합}
앞서 구한 (\ref{area})의 관계가 \(l\Ray{OA}+m\Ray{OB}+n\Ray{OC}=\Ray{O}\)에서 나왔음을 생각하면, 당연히 \(s_A \Ray{OA}+s_B \Ray{OB}+s_C \Ray{OC}\)의 결과는 영벡터이다... 라고 하면서 끝내고 싶지만 그래도 증명을 해보자.

\(s_B:s_A=\triangle OCA:\triangle OBC=\triangle RCA:\triangle RCB=\Seg{AR}:\Seg{BR}\)이고, \(\triangle ABC:\triangle OAB=(s_A+s_B+s_C):s_C=\Seg{CR}:\Seg{OR}\)이므로, 다음이 성립함을 알 수 있다.
\begin{align}\label{OR}\Ray{OR}=\frac{s_A\Ray{OA}+s_B\Ray{OB}}{s_A+s_B}\end{align}
\begin{align}\label{CO}\Seg{CO}:\Seg{OR}=(s_A+s_B):s_C\end{align}
(\ref{OR}), (\ref{CO})의 관계를 사용하면 다음과 같은 식을 얻는다.
\[\Ray{OR}-\frac{s_C\Ray{CO}}{s_A+s_B}=\Ray{OR}+\frac{s_C\Ray{OC}}{s_A+s_B}=\frac{s_A\Ray{OA}+s_B\Ray{OB}}{s_A+s_B}+\frac{s_C\Ray{OC}}{s_A+s_B}=\Ray{O}\]
이 식을 변형하면 주어진 식 \(s_A \Ray{OA}+s_B \Ray{OB}+s_C \Ray{OC}\)의 결과는 영벡터임을 알 수 있다.

\end{document}
