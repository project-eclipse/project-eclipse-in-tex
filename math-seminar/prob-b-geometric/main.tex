\documentclass{scrartcl}
\usepackage[left=2cm, right=2cm, top=2cm, bottom=2cm]{geometry}
\usepackage{amsmath}
\usepackage{amssymb}
\usepackage{amsthm}
\usepackage{blindtext}
\usepackage{datetime}
\usepackage{fontspec}
\usepackage{float}
\usepackage{graphicx}
\usepackage{hyperref}
\usepackage{mathrsfs}
\usepackage{mathtools}
\usepackage{pgf,tikz,pgfplots}

\usepackage[headsepline]{scrlayer-scrpage}
\newcommand{\doctitle}{수학세미나 문제 B 기하학적 풀이}
\clearpairofpagestyles
\ohead{\thepage}
\ihead{\doctitle}

\hypersetup{
    colorlinks=true,
    linkcolor=black,
    urlcolor=blue,
}
\pgfplotsset{compat=1.15}
\usetikzlibrary{arrows}
\newtheorem{theorem}{Theorem}

\setmainfont{Noto Serif CJK KR}
\setsansfont{NanumSquare}
\newfontfamily\sectionfont{NanumSquare Bold}
\addtokomafont{section}{\sectionfont\mdseries}
\addtokomafont{subsection}{\sectionfont\mdseries}
\addtokomafont{subsubsection}{\sectionfont\mdseries}
\addtokomafont{paragraph}{\sectionfont\mdseries}
\addtokomafont{title}{\sectionfont\mdseries}
\addtokomafont{author}{\sectionfont}
\title{\doctitle}
\author{Project Eclipse (손량)}
\date{Last compiled on: \today, \currenttime}

\newcommand{\Lim}[1]{\lim_{#1\to\infty}}
\newcommand{\para}{\mathbin{\!/\mkern-5mu/\!}}
\newcommand{\Seg}[1]{\overline{#1}}
\newcommand{\Line}[1]{\overleftrightarrow{#1}}
\newcommand{\Ray}[1]{\overrightarrow{#1}}
\newcommand{\infsum}[1]{\sum^\infty_{#1}}
\newcommand{\mat}[3]{\begin{pmatrix}
#1_{11} & #1_{12} & \cdots & #1_{1#3} \\
#1_{21} & #1_{22} & \cdots & #1_{2#3} \\
\vdots & \vdots & \ddots & \vdots \\
#1_{#2 1} & #1_{#2 2} & \cdots & #1_{#2#3} \\
\end{pmatrix}}
\newcommand{\un}[1]{\ensuremath{\ \mathrm{#1}}}

\begin{document}
\maketitle

\section{문제의 풀이}
B의 3번 문제를 기하학적인 방법을 사용하여 풀어보자. 관심법을 쓰면, 주어진 식을 직선과 점 사이 거리 공식처럼 만들 수 있을 것 같다.
\[\frac{x-y-5}{x+y-3}=\frac{\dfrac{x-y-5}{\sqrt{1^2+(-1)^2}}}{\dfrac{x+y-3}{\sqrt{1^2+1^2}}}\]
게다가 \(y=\pm x\pm\sqrt{5}\) (복부호 동순 아님)이 타원의 접선임을 생각하면, \(|x+y|,|x-y|\leq\sqrt{5}\)이고, 다음과 같이 절댓값을 붙여도 무방하며 그냥 점과 직선 사이 거리로 생각해도 된다.
\begin{align}\label{slope}\frac{x-y-5}{x+y-3}=\frac{\dfrac{|x-y-5|}{\sqrt{1^2+(-1)^2}}}{\dfrac{|x+y-3|}{\sqrt{1^2+1^2}}}=\frac{\text{\(x-y-5=0\)과 \((x,y)\)의 거리}}{\text{\(x+y-3=0\)과 \((x,y)\)의 거리}}\end{align}
한편, 직선 \(x-y-5=0\)과 \(x+y-3=0\)의 교점은 앞서 등장한 \((4,-1)\)이고\footnote{그렇다! 세상은 그래도 아름다웠던 것이다...}, 두 직선은 서로 수직이다. 이 문제는 두 직선이 수직\footnote{수직이 아니었다면 벡터의 기저 분해를 시도할 수도 있기는 한데, 그럴 바에는 CBS 부등식을 사용하는 것이 나을 듯 하다.} 것이 중요했는데, 그냥 두 직선으로 `새로운 좌표축'을 정할 수 있기 때문이다.
\begin{figure}[H]
\centering
\definecolor{qqwuqq}{rgb}{0,0.39215686274509803,0}
\definecolor{uuuuuu}{rgb}{0.26666666666666666,0.26666666666666666,0.26666666666666666}
\begin{tikzpicture}[line cap=round,line join=round,>=triangle 45,x=1.5cm,y=1.5cm]
\clip(-3,-1.1) rectangle (3,2);
\draw[line width=1pt,color=qqwuqq,fill=qqwuqq,fill opacity=0.10000000149011612] (2.389020181518927,-0.7494503531228769) -- (2.1384705346418036,-0.7494503531228769) -- (2.1384705346418036,-1) -- (2.389020181518927,-1) -- cycle; 
\draw [shift={(-2,-1)},line width=1pt,color=qqwuqq,fill=qqwuqq,fill opacity=0.10000000149011612] (0,0) -- (0:0.3543307086614173) arc (0:45:0.3543307086614173) -- cycle;
\draw [samples=50,rotate around={0:(0,0)},xshift=0cm,yshift=0cm,line width=1pt,domain=-8:8)] plot (\x,{(\x)^2/2/2});
\draw [line width=1.5pt,domain=-3:3] plot(\x,{(-1-0*\x)/1});
\draw [->,line width=1pt] (-2,-1) -- (0,1);
\draw [->,line width=1pt] (-2,-1) -- (0.83,-1);
\draw [->,line width=1pt] (0,1) -- (2.389020181518927,1.4268543569261816);
\draw [->,line width=1pt] (2.389020181518927,1.4268543569261816)-- (2.389020181518927,-1);
\draw [->,line width=1pt] (-2,-1) -- (2.3890201815189265,1.4268543569261816);
\begin{scriptsize}
\draw [fill=black] (0,1) circle (2pt);
\draw[color=black] (0.09448818897637795,1.1802184466019414) node {$B$};
\draw [fill=black] (-2,-1) circle (2pt);
\draw[color=black] (-2.1,-0.8161407766990291) node {$O$};
\draw [fill=black] (0.83,-1) circle (2pt);
\draw[color=black] (0.9212598425196851,-0.8161407766990291) node {$A$};
\draw[color=black] (-1.4,-0.1) node {$\vec{b}$};
\draw[color=black] (-0.5433070866141732,-0.8) node {$\vec{a}$};
\draw [fill=black] (2.389020181518927,1.4268543569261816) circle (2pt);
\draw[color=black] (2.3,1.6134708737864072) node {$P$};
\draw [fill=uuuuuu] (2.389020181518927,-1) circle (2pt);
\draw[color=uuuuuu] (2.6,-0.833131067961165) node {$H$};
\draw[color=black] (0.20078740157480315,0.4) node {$\vec{p}$};
\draw[color=qqwuqq] (-1.5,-0.88) node {$\theta$};
\end{scriptsize}
\end{tikzpicture}

\end{figure}
또한, (\ref{slope})의 식은 \(x'\)축과 \(y'\)축을 기준으로 했을 때 \(O'\)을 지나는 직선의 기울기와 같다고 볼 수 있고, 이때 기울기의 최대와 최소는 위 그림에 표시해 놓았다. 최소일 때에는 \(xy\) 죄표계에서 직선이 \(y=-1\)일 때이고, 최대일 때에는 \(y=-\dfrac{2}{3}(x-4)-1\)이다. 이제 이 \(xy\) 좌표계의 직선의 기울기를 \(x'y'\) 좌표계의 직선 기울기로 바꾸어 주어야 하는데, 여기에는 탄젠트 덧셈정리를 사용하면 된다. 우선, 최소일 때에는 굳이 계산을 열심히 할 필요 없이, \(y=x\)임을 알 수 있다. 최대일 때 직선이 \(x\)축의 음의 방향과 이루는 각의 크기가 \(\theta\)일 때,
\[\tan\left(\theta+\frac{\pi}{4}\right)=\frac{\tan\theta+\tan\frac{\pi}{4}}{1-\tan\theta\tan\frac{\pi}{4}}=\frac{\frac{2}{3}+1}{1-\frac{2}{3}\cdot1}=5\]
따라서 구하는 최소는 1이고, 최대는 5이다.

\section{왜 이렇게 풀었는가?}
문제를 처음 봤을 때 직선과 점 사이 거리 공식이 생각났고, 거기에서부터 문제풀이를 시작했다.

좌표축을 기울이는 아이디어는 심화 미분적분학 교재인 미적분학 1+의 각주를 해결하는 과정에서 알게 되었다. 미적분학 1+ 9.1에서 데카르트 곡선의 매개화를 다루는데 책에 나온 방법에서는 \(t\neq -1\)일때의 매개화만 가능하다. 그 책이 언제나 그러듯 각주에 ``실수 전체 구간에서 정의된 매개화도 가능하다'' 라고 써있는데\footnote{``도발하는데''라고 읽자.}, 요구하는 대로 매개화를 해 보았다. \(y=tx\)꼴의 직선의 방정식이 나타내지 못하는 직선은 수직선이라는 것에 착안해, 좌표축 자체를 기울여 \(xy\) 좌표계 에서의 기울기가 -1이면 기울인 좌표계에서는 수직선이 되도록 기울이면 되겠다는 생각이 들었다.

\end{document}
