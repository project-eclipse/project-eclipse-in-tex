\documentclass{scrartcl}

\usepackage[margin=0.5in]{geometry}
\usepackage{amsmath}
\usepackage{amssymb}
\usepackage{cancel}
\usepackage{epigraph}
\usepackage{fontspec}
\usepackage{graphicx}
\usepackage{hyperref}
\usepackage{mathrsfs}
\usepackage{pgf,tikz,pgfplots}
\usepackage{wrapfig}

\hypersetup{
    colorlinks=true,
    linkcolor=black,
    urlcolor=blue,
}
\pgfplotsset{compat=1.15}
\usetikzlibrary{arrows}

\setmainfont{Noto Serif CJK KR}
\setsansfont{Noto Sans CJK KR}
\newfontfamily\sectionfont{Noto Sans CJK KR Medium}
\addtokomafont{section}{\sectionfont\mdseries}
\addtokomafont{subsection}{\sectionfont\mdseries}
\addtokomafont{subsubsection}{\sectionfont\mdseries}
\addtokomafont{paragraph}{\sectionfont\mdseries}
\addtokomafont{title}{\sectionfont\mdseries}
\addtokomafont{author}{\sectionfont}
\title{미적분학 개념 정리}
\author{Project Eclipse (손량 X 임지안)}
\date{}

\newcommand{\Lim}[1]{\lim_{#1\to\infty}}
\newcommand{\para}{\mathbin{\!/\mkern-5mu/\!}}
\newcommand{\Seg}[1]{\overline{#1}}
\newcommand{\Line}[1]{\overleftrightarrow{#1}}
\newcommand{\Ray}[1]{\overrightarrow{#1}}
\newcommand{\infsum}[1]{\sum^\infty_{#1}}
\newcommand{\mat}[3]{\begin{pmatrix}
#1_{11} & #1_{12} & \cdots & #1_{1#3} \\
#1_{21} & #1_{22} & \cdots & #1_{2#3} \\
\vdots & \vdots & \ddots & \vdots \\
#1_{#2 1} & #1_{#2 2} & \cdots & #1_{#2#3} \\
\end{pmatrix}}

\begin{document}
\maketitle

\section{수열의 극한}
\subsection{역수 정리}
수렴하는 수열 \(a_n\)과 무한대로 발산하는 수열\footnote{정확히는, ``절댓값이 무한대로 발산하는 수열''이다.} \(b_n\)에 대해 다음이 성립한다.
\[\lim_{n \to \infty} {\frac{a_n}{b_n}}=0\]
정말 별 거 없어 보이지만, 서술형 극한 문제를 풀 때 `\(b_n\)으로 나눠도 되나?' 하는 의문이 들 때 상당히 유용한 정리이다.

\subsection{수열의 일반항과 부분합의 관계}
등차수열의 부분합 \(S_n\)의 최고차항이 \(an^2\)일 때, 일반항의 최고차항, 즉 \(n\)차항은 \(2an\)이다. 예를 들어, 어떤 등차수열 \(\{a_n\}\)의 부분합 \(S_n\)이 다음과 같다고 하자.
\[
S_n = 4n^2 - n
\]
이때 \(a_n\)의 최고차항은 \(8n\)이고, \(S_1 = a_1\)임을 이용하면 일반항을 구할 수 있다.
\[
a_n = 8n - 5
\]
이를 다르게 해석하면, 수열 \(a_n\) 일반항의 최고차항은 부분합 \(S_n\)의 최고차항을 \(n\)에 대해 미분한 것이라고 할 수 있다. 이는 \(a_n\)이 \(n\)차일 때 일반적으로 성립한다.

\subsection{수열의 닫힌 연산}
어떤 집합에서의 연산을 수행한 결과가 언제나 그 집합에 존재하면, 집합은 그 연산에 대해 닫혀 있다고 한다. 예를 들어, 정수 집합은 덧셈과 곱셈에 대해 닫혀 있다고 할 수 있다. 정수 두 개를 더하거나 곱해도 여전히 정수이기 때문이다. 여기에서는 수열에 대해 닫힌 연산을 살펴볼 것이다.

\subsubsection{등차수열의 경우}
\begin{wrapfigure}[11]{l}{7.5cm}
\bgroup
\def\arraystretch{1.5}
\begin{tabular}{|lll|}
\hline
연산                  & 닫힘 여부 & 공차            \\ \hline
덧셈: \(c_n=a_n+b_n\) & 닫혀 있음 & \(d_1 + d_2\) \\
뺄셈: \(c_n=a_n-b_n\) & 닫혀 있음 & \(d_1 - d_2\) \\
상수배: \(c_n=ka_n\)   & 닫혀 있음 & \(kd_1\)      \\
합성: \(c_n=a_{b_n}\) & 닫혀 있음 & \(d_1d_2\)    \\ \hline
\end{tabular}
\egroup
\end{wrapfigure}
표에 있는 두 등차수열 \(a_n\)과 \(b_n\)은 다음과 같이 정의된다.
\[
a_n:=d_1n+p_1,\,b_n:=d_2n+p_2
\]
또한, 상수배는 수업 시간에는 다루지 않았지만 논의의 완결성을 위해서 추가하였다. 상수배 조건까지 이용하면, 다음과 같이 등차수열을 일차결합해도 여전히 등차수열이다.\footnote{또한, 상수배를 추가하면 뺄셈을 굳이 다룰 필요가 사실 없어진다. 뺄셈도 일차결합으로 해석할 수 있기 때문이다.}
\[
c_n=ka_n+la_n
\]
이때 수열의 공비는 \(kd_1+ld_2\)이다.

\subsubsection{등비수열의 경우}
\begin{wrapfigure}[11]{l}{8.5cm}
\bgroup
\def\arraystretch{1.5}
\begin{tabular}{|lll|}
\hline
연산                         & 닫힘 여부 & 공비             \\ \hline
덧셈, 뺄셈: \(c_n=a_n\pm b_n\) & 제한적   & \(r_1\)        \\
상수배: \(c_n=ka_n\)          & 닫혀 있음 & \(r_1\)        \\
곱셈: \(c_n=a_nb_n\)         & 닫혀 있음 & \(r_1r_2\)     \\
나눗셈: \(c_n=a_n\div b_n\)   & 닫혀 있음 & \(r_1/r_2\)    \\
역수: \(c_n={a_n}^{-1}\)     & 닫혀 있음 & \({r_1}^{-1}\) \\
등차수열 합성: \(c_n=a_{pn+q}\)  & 닫혀 있음 & \({r_1}^p\)    \\ \hline
\end{tabular}
\egroup
\end{wrapfigure}
표에 있는 두 등비수열 \(a_n\)과 \(b_n\)은 다음과 같이 정의한다.
\[
a_n:=a{r_1}^{n-1},\,b_n:=b{r_2}^{n-1}
\]
덧셈과 뺄셈은 공비가 같을 때만 닫혀 있다. 여기에서도 상수배를 추가했고, 공비가 같을 때에 한해 선형결합을 해도 여전히 등비수열이다.
\[
c_n=ka_n+la_n
\]
이때 수열의 공비는 \(r_1\)이다.

\subsection{기우수열}
기우수열이란, \(1, 2, 1, 3, 1, 4,\cdots\)처럼 홀수 항과 짝수 항의 규칙이 다른 수열을 의미한다. \(a_n\)과 \(a_{n+1}\)과 같이 이웃한 두 항 간의 관계식이 주어지는 수열이라면 기우수열인지 의심해 볼 필요성이 있다.\\[1\baselineskip]
기우수열이나 기우수열의 합을 다룰 때는 각각의 규칙으로 쪼개서는 안 되며, 오히려 다음과 같이 2개의 항을 모아 더하는 것이 편리하다.
\[
\infsum{n=0}a_n =\infsum{n=1}(a_{2n-1}+a_{2n})
\]
기우수열의 자매품으로는 3번째, 4번째, 혹은 m번째 항마다 같은 규칙이 적용되는 `주기수열'이 있다. 주기수열 역시 각각의 항으로 나누어 생각하지 말고, 다음과 같이 묶어서 더해야 한다.
\[
\infsum{n=0}a_n =\infsum{n=1}(a_{kn-k+1}+a_{kn-k+2}+\cdots+a_{kn})
\]
\subsection{\texorpdfstring{\(a_{n+1} = pa_n+q\)}꼴의 점화식}
공식적으로 점화식은 교육과정에서 사라졌지만, 이 점화식만큼은 유일하게 남아 문제로 출제될 가능성이 있다. `공식'적 풀이는 귀납적 추론\footnote{``숫자 찍어서 맞히기''라고 읽으면 된다.}을 이용하는 것이지만, 현실적으로 시험 상황에서 이렇게 풀기는 쉽지 않다. 따라서 이런 점화식을 보게 된다면, 다음 풀이를 익혀두도록 하자.
\\[1\baselineskip]
점화식 \(a_{n+1} = pa_n+q\)를 다음과 같이 변형해 보자.
\[
a_{n+1}-\frac{q}{1-p} = p\times(a_n-\frac{q}{1-p})
\]
새로운 수열 \(\{a_n-\frac{q}{1-p}\}\)는 공비가 p인 등비수열이다. 따라서 \(a_n\)의 수렴조건은 \(|p|<1\)임을 알 수 있다.
\\[1\baselineskip]
다시 말해, 문제에서 \(|p|<1\)인 점화식을 준다면 이 수열이 수렴한다는 사실을 알 수 있다. 이때 \(a_n\)의 수렴값을 \(\alpha\)라 하면, \(a_{n+1}\)도 \(\alpha\)로 수렴하기 때문에\footnote{수열에서 유한 개의 항을 제거하더라도 극한은 변하지 않는다. 그리고 수열 \(a_{n+1}\)은 수열 \(a_n\)에서 첫 항을 제거한 것이다.} \(\alpha = p\alpha + q\)라는 식을 세울 수 있다. 따라서 \(\alpha\)는 \(\frac{q}{1-p}\)임을 쉽게 구할 수 있다.

\subsection{등차수열의 합을 나타내는 여러 가지 방법}
등차수열의 합은 \(\text{(개수)}\times\text{(등차중항)}\)라는 관점에서 보았을 때, 다음과 같이 나타낼 수도 있다. (\(a_n:=d(n-1)+a\)라고 두었다.)
\begin{align*}
S_n=&n\left(\frac{a_1+a_n}{2}\right) \\
=&n\left(\frac{a+\{d(n-1)+a\}}{2}\right)=n\left(\frac{2a+(n-1)d}{2}\right)
\end{align*}

\subsection{등비수열의 곱}
등차수열의 합을 \(\text{(개수)}\times\text{(등차중항)}\)으로 나타낸다는 아이디어를 그대로 이용해 보자. \(a_n\)이 등비수열이라면, 다음과 같이 합을 나타낼 수 있다.
\[
p_n=\prod^n_{k=1}a_k=\text{(등비중항)}^n=\left(\sqrt{a_1a_n}\right)^n
\]
등차수열의 부분합에서 일반항을 구하듯이 등비수열의 ``부분곱''에서 일반항을 구할 수도 있다. 물론 등차수열처럼 \(n\geq2\)라는 조건이 붙는다.
\[
a_n=\frac{p_n}{p_{n-1}}
\]

\subsection{소거되는 합 (Telescoping sum)}
다음과 같은 형태의 합은 처음과 나중 항만 남는다. \(\Lim{n} a_n=0\)인 경우 남은 항들 중 뒤쪽에 있는 것은 무시 가능하다.
\[
\sum^n_{k=1}(a_k-a_{k+1})=(a_1-\cancel{a_2})+(\cancel{a_2}-\cancel{a_3})+\cdots+(\cancel{a_n}-a_{n+1})=a_1-a_{n+1}
\]
이러한 형태의 소거형은 주로 부분 분수나, \((-1)^n\)과 같은 계수가 붙어서 부호가 번갈아 가며 나오는 기우수열의 합에서 유용하다.
\subsubsection{2칸 Telescoping 구분}
다음 두 급수는 형태가 상당히 비슷하다. 그러나 하나는 첫 항만 살아남고, 다른 하나는 2개의 항이 살아남는다. 왜일까?
\[
\infsum{n=1}\frac{1}{2n-1}-\frac{1}{2n+1} = (\frac{1}{1}-\cancel{\frac{1}{3}})+(\cancel{\frac{1}{3}}-\cancel{\frac{1}{5}})+(\cancel{\frac{1}{5}}-\cancel{\frac{1}{7}})+\cdots = 1
\]
\[
\infsum{n=1}\frac{1}{n}-\frac{1}{n+2} = (\frac{1}{1}-\cancel{\frac{1}{3}})+({\frac{1}{2}}-\cancel{\frac{1}{4}})+(\cancel{\frac{1}{3}}-\cancel{\frac{1}{5}})+\cdots = 1+\frac{1}{2} = \frac{3}{2}
\]
두 번째 급수는 1항과 3항, 2항과 4항, 3항과 5항\(\cdot\) 이렇게 한 칸 건너의 항과 소거가 일어나는 ``2칸 Telescoping'' 급수이다. 그러나 첫 급수는 한 칸씩 건너뛰지 않고, 인접한 항끼리 바로 소거가 일어난다. 즉, 숫자만 \(2n\)일 뿐, \(\infsum{n=1}\frac{1}{n(n+1)}\)과 같은 ``1칸 텔레스코핑'' 급수라는 것이다. 이처럼 소거되는 합 문제를 풀 때는 앞에 몇 개의 항이 남는지 꼭 검토하도록 하자.

\subsubsection{소거되는 곱}
소거되는 합과 마찬가지로 소거되는 곱도 존재한다.
\[
\prod^n_{k=1}\frac{a_k}{a_{k+1}}=\frac{a_1}{\cancel{a_2}}\times\frac{\cancel{a_2}}{\cancel{a_3}}\times\cdots\times\frac{\cancel{a_n}}{a_{n+1}}=\frac{a_1}{a_{n+1}}
\]
가끔 이러한 소거되는 곱이 자연상수 \(e\)가 결과로 나오는 극한을 계산할 때 사용되기도 한다.

\subsection{직각삼각형의 내접원}
직각삼각형의 내접원의 반지름 구하기는 극한 문제에서 가장 자주 등장하는 평면도형의 성질 중 하나이다. 세 변의 길이가 \(a\), \(b\), \(c\)이고, 빗변의 길이가 \(a\)인 직각삼각형의 내접원의 반지름을 구하는 공식은 다음과 같이 2가지가 있다.
\begin{align*}
r&= \frac{b+c-a}{2} \tag{길이 이용} \\
&= \frac{bc}{a+b+c} \tag{넓이 이용}
\end{align*}

\subsubsection{증명}
\begin{wrapfigure}[9]{l}{4cm}
\vspace{-25pt}
\centering
\definecolor{ududff}{rgb}{0.30196078431372547,0.30196078431372547,1}
\definecolor{xdxdff}{rgb}{0.49019607843137253,0.49019607843137253,1}
\definecolor{uuuuuu}{rgb}{0.26666666666666666,0.26666666666666666,0.26666666666666666}
\begin{tikzpicture}[line cap=round,line join=round,>=triangle 45,x=1cm,y=1cm]
\clip(-0.5,-0.2) rectangle (4,5);
\draw [line width=1pt] (0,4)-- (0,0);
\draw [line width=1pt] (0,0)-- (3,0);
\draw [line width=1pt] (3,0)-- (0,4);
\draw [line width=1pt] (1,1) circle (1cm);
\begin{scriptsize}
\draw[color=uuuuuu] (-0.2,-0.1) node {$A$};
\draw[color=uuuuuu] (0,4.2) node {$B$};
\draw[color=uuuuuu] (3.2,0) node {$C$};
\draw [fill=ududff] (1,1) circle (1.5pt);
\draw[color=ududff] (1.0704012578134492,1.184027345985) node {$I$};
\draw [fill=uuuuuu] (1.8,1.6) circle (1.5pt);
\draw[color=uuuuuu] (1.9,1.8) node {$D$};
\draw [fill=uuuuuu] (0,1) circle (1.5pt);
\draw[color=uuuuuu] (-0.2,1) node {$F$};
\draw [fill=uuuuuu] (1,0) circle (1.5pt);
\draw[color=uuuuuu] (1,-0.2) node {$E$};
\end{scriptsize}
\end{tikzpicture}
\end{wrapfigure}
왼쪽 그림에서, 식에 나온 문자들을 다음과 같이 계산할 수 있다.
\[
a=\Seg{BD}+\Seg{DC},\,b=\Seg{AE}+\Seg{EC},\,c=\Seg{BF}+\Seg{FA}
\]
이때, 직각삼각형이므로 \(\Seg{FI}\para\Seg{AE},\,\Seg{EI}\para\Seg{AF}\)이고, 여기서 내접원의 반지름을 계산해 길이를 이용한 공식을 유도할 수 있다.
\begin{align*}
r=&\Seg{FI}=\frac{\Seg{FA}+\Seg{AE}}{2}=\frac{\left(\Seg{BF}+\Seg{FA}\right)+\left(\Seg{AE}+\Seg{EC}\right)-\left(\Seg{BD}+\Seg{DC}\right)}{2}\\
=&\frac{b+c-a}{2}
\end{align*}
넓이를 이용한 공식은 중학교 때 배운\footnote{``이미 잊어버린'' 이라고 읽으면 된다.} 내심 관련 공식에서 알 수 있다.
\[
S=\frac{r}{2}(a+b+c)
\]

\subsection{구간함수에서의 연속}
함수 \(f(x)\)가 다음과 같은 형태이고, \(g\)와 \(h\)가 실수 전체에서 정의역에서 연속이라고 가정하자.\footnote{물론 실수 전체가 아니여도 구간함수에서 자신이 주어지는 범위에서만 연속이여도 상관은 없다.}
\[
f(x)=\begin{cases}
g(x) & (x < a) \\
h(x) & (x \geq a)
\end{cases}
\]
이때 \(f\)가 정의역에서 연속이기 위해서는 다음이 성립해야 한다.
\begin{align}\label{continuity_condition}
\lim_{x\to a-} f(x) = \lim_{x\to a+} f(x) = f(a)
\end{align}
여기에서, 초기에 한 가정을 이용하면 극한값을 바로 계산할 수 있다.
\[
\lim_{x\to a-} f(x) = \lim_{x\to a-} g(x) = g(a)
\]
\[
\lim_{x\to a+} f(x) = \lim_{x\to a+} h(x) = h(a)
\]
따라서 (\ref{continuity_condition})의 식은 다음과 같이 쓸 수 있다.
\[
g(a) = h(a) = f(a)
\]
결론적으로, 구간함수가 연속인지 판별하려면 ``경계''만 보면 된다고 할 수 있다.\footnote{수학에서는 이 ``경계''가 중요하다고 선생님께서 말하신 것을 기억할 것이다.}

\subsection{불연속점의 종류}
어떤 함수가 \(f(x)\)가 \(x=a\)에서 불연속일 때, 다음과 같은 종류들이 존재한다.\footnote{사실 여기에서 무한진동이 빠졌는데, 적절한 예를 찾기 어려웠다.}
\begin{figure}[!h]
\minipage{0.32\textwidth}
    \centering
    \begin{tikzpicture}
    \draw[line width=1pt] (-2, -2) -- (2, 2);
    \draw[line width=1pt,dash pattern=on 1.5pt off 1.5pt] (0, 0) -- (0, 1.5);
    \filldraw[color=black, fill=white] (0,0) circle (0.075);
    \filldraw[fill=black] (0,1.5) circle (0.075);
    \end{tikzpicture}
    \caption{Removable}
    \label{fig:my_label}
\endminipage\hfill
\minipage{0.32\textwidth}
    \centering
    \begin{tikzpicture}
    \draw[line width=1pt] (-2.4, -2.4) -- (0, 0);
    \draw[line width=1pt] (0, 1.5) -- (2, 0);
    \draw[line width=1pt,dash pattern=on 1.5pt off 1.5pt] (0, 0) -- (0, 1.5);
    \filldraw[color=black, fill=white] (0,0) circle (0.075);
    \filldraw[fill=black] (0,1.5) circle (0.075);
    \end{tikzpicture}
    \caption{Jump}
    \label{fig:my_label}
\endminipage\hfill
\minipage{0.32\textwidth}
    \centering
    \definecolor{qqwuqq}{rgb}{0,0.39215686274509803,0}
    \begin{tikzpicture}[line cap=round,line join=round,>=triangle 45,x=1cm,y=1cm]
    \begin{axis}[
    x=1cm,y=1cm,
    axis lines=middle,
    xmin=-2,
    xmax=2,
    ymin=-2,
    ymax=2,
    ticks=none,]
    \clip(-2,-2) rectangle (2,2);
    \draw[line width=1.5pt,color=black,smooth,samples=100,domain=-2:-0.4] plot(\x,{1/(\x)});
    \draw[line width=1.5pt,color=black,smooth,samples=100,domain=0.4:2] plot(\x,{1/(\x)});
    \end{axis}
    \end{tikzpicture}
    \caption{발산}
    \label{fig:my_label}
\endminipage\hfill
\end{figure}
이들 중 Removable\footnote{RM(아이돌이 아니라 파일삭제 명령어)라고 부르기도 한다.}과 Jump는 제1종 불연속점, 발산과 무한진동은 제2종 불연속점으로 분류한다.

\subsection{곱함수의 연속}
\(x=a\)에서 \(f(x)\)는 연속이고, \(g(x)\)는 제1종 불연속점을 가진다고 하자. 좌극한과 우극한이 존재하므로 다음과 같이 놓을 수 있을 것이다.
\[
\lim_{x\to a-}g(x)=l,\,\lim_{x\to a+}g(x)=r
\]
여기에서, \(f(x)g(x)\)가 \(x=a\)에서 연속이기 위해서는 \(f(a)=0\)이어야 한다.

\subsection{제곱근}
n이 무한대로 갈 때, \(\sqrt{an^2+bn+c}\) 꼴의 식에서 일반적으로 c를 어떤 숫자로 바꾸든 극한값에 영향이 가지 않는다. 무슨 말인지 감이 잘 안 온다면, 다음 예제를 통해 알아보자.
\begin{align*}
&\lim_{n \to \infty} \left(\sqrt{n^2+4n+3}-\sqrt{n^2+2n-6}\right) \\
&= \lim_{n \to \infty} \left(\sqrt{n^2+4n+4}-\sqrt{n^2+2n+1}\right) \\
&= \lim_{n \to \infty} ((n+2)-(n+1)) = 1
\end{align*}
원래대로라면, 이 극한을 계산하기 위해서는 복잡한 유리화 과정을 거쳐야 한다. 그러나 상수항을 몇 번 만지더라도 극한값에 영향이 가지는 않기 때문에, 위와 같이 손쉽게 풀 수도 된다.
\subsection{도형의 극한 문제를 푸는 방법}
\begin{wrapfigure}[5]{l}{6cm}
\vspace{-10pt}
\bgroup
\def\arraystretch{1.5}
\begin{tabular}{|lll|}
\hline
개수     & 길이의 합               & 넓이의 합                \\ \hline
1개     & \(\dfrac{a}{1-r}\)  & \(\dfrac{a}{1-r^2}\)  \\
\(m\)개 & \(\dfrac{a}{1-mr}\) & \(\dfrac{a}{1-mr^2}\) \\ \hline
\end{tabular}
\egroup
\end{wrapfigure}
도형의 극한 문제는 99\% 확률로 닮음을 활용한 등비급수 계산이다.\footnote{99\%라고 한 것은 무조건 그렇지는 않기 때문이다. 이를테면 등비급수가 아니라 Telescoping sum일수도 있다. 물론 그런 문제를 본적이 없긴 하지만...} 이런 유형의 문제를 풀 때는  지금까지 배웠던 평면도형의 성질을 이용하여 `초항'과 `닮음비'를 구하면 된다. 그리고 구하고자 하는 합이 길이의 합인지 혹은 넓이의 합인지, 다음 그림에서 생기는 닮은 도형의 개수가 1개인지, 여러 개 (\(m\)개)인지에 따라 표에 나온 공식대로 적절히 대입하면 된다.

\section{미분}
\subsection{극한의 기본 성질: "미리 보내기" (Revisited)}
미분 단원에서 다루는 초월함수의 극한에서 볼 수 있는 형태로 다음과 같은 것이 있다.
\[
\lim_{x\to a}\frac{f(x)+g(x)}{h(x)},\,\lim_{x\to a}f(x)=L
\]
이때, 다음과 같은 식은 매우 제한적인 경우에만 성립한다.
\[
\lim_{x\to a}\frac{f(x)+g(x)}{h(x)}=\frac{L+\lim_{x\to a}g(x)}{\lim_{x\to a}h(x)}
\]
경우를 잘 나누어서 어떤 경우에 가능한지 보자.

\subsubsection{\texorpdfstring{\(\lim_{x\to a}h(x)=\infty\)}인 경우}
``노잼인 경우''이다. 다음과 같이 나눌 수 있다.
\begin{align*}
    \lim_{x\to a}g(x)=M&\longrightarrow\frac{C}{\infty}\text{꼴이므로 }0 \\
\lim_{x\to a}g(x)=\infty&\longrightarrow\frac{\infty}{\infty}\text{꼴, 계산해 봐야 함}
\end{align*}

\subsubsection{\texorpdfstring{\(\lim_{x\to a}h(x)=0\)}인 경우}
``짜증나는 경우''이다. 다음과 같이 나눌 수 있다.
\begin{align*}
\lim_{x\to a}g(x)\neq -L&\longrightarrow\frac{C}{0}\text{꼴이므로 }\infty \\
\lim_{x\to a}g(x)=-L=0&\longrightarrow\text{0인수 개수 비교} \\
\lim_{x\to a}g(x)=-L\neq 0&\longrightarrow\text{쪼갤 수 없음}
\end{align*}
0인수의 개수를 비교할 때, \(f\)의 0인수가 \(h\) 0인수 개수 이상이면 쪼개도 되고, 그렇지 않으면 쪼개면 안 된다. 여기 0인수라는 것은 \(x=a\)를 대입할 때 0이 되는 인수의 개수를 의미한다.
\[
\lim_{x\to a}\frac{f(x)}{g(x)}=\lim_{x\to a}\frac{(x-a)^2(x-b)}{(x-a)}
\]
예를 들어 위 식에서 분자인 \(f(x)\)의 0인수 개수는 2개이고, 분모인 \(g(x)\)의 0인수 개수는 1개이다.

\subsubsection{\texorpdfstring{\(\lim_{x\to a}h(x)=N\)}인 경우}
또 다른 ``노잼인 경우''이다. 다음과 같이 극한값을 계산하면 된다.
\[
\frac{L+\lim_{x\to a}g(x)}{N}
\]

\subsection{테일러 급수를 이용한 극한 계산}
미적분 단원에서 다루는 몇 가지 함수들은 다음과 같이 거듭제곱급수로 나타낼 수 있다.\footnote{마지막 \(\ln\)의 거듭제곱급수 빼고는 수렴반경이 무한이다. 물론 여기에서는 0으로 가는 극한만 다루므로 상관은 없다.}
\begin{align*}
\sin x=\infsum{n = 0}(-1)^n\frac{x^{2n+1}}{(2n+1)!}&=x-\frac{x^3}{3!}+\frac{x^5}{5!}-\cdots \\
\tan x=\infsum{n = 1}A_{2n-1}\frac{x^{2n-1}}{(2n-1)!}&=x+\frac{1}{3}x^3+\frac{2}{15}x^5+\cdots \\
e^x=\infsum{n=0}\frac{x^n}{n!}&=1+x+\frac{x^2}{2!}+\cdots \\
\ln (1+x)=\infsum{n=1}\frac{(-1)^{n+1}x^n}{n}&=x-\frac{x^2}{2}+\frac{x^3}{3}-\cdots
\end{align*}
\(x=0\) 근처에서는 고차항들이 0에 빠르게 수렴함을 이용하면 다음과 같은 근사가 가능하다.
\[
\sin x\approx x,\,\tan x\approx x,\,e^x\approx 1+x,\,\ln(1+x)\approx x
\]
이러한 근사는 초월함수의 극한 식과 잘 일치한다.
\[
\lim_{x\to0}\frac{\sin x}{x}=1,\,\lim_{x\to0}\frac{\tan x}{x}=1,\,\lim_{x\to0}\frac{e^x-1}{x},\,\lim_{x\to0}\frac{\ln(1+x)}{x}=1
\]
이러한 근사를 사용할 때, 주의해야 하는 경우들이 있다. 이러한 경우는 함수들의 성질을 이용해 우선 수렴하는 꼴로 만들어서 계산해야 한다.
\begin{align*}
\lim_{x\to0}\frac{\ln(1+x)+\ln(1-x)}{x^2}&\neq\lim_{x\to0}\frac{x - x}{x^2}=0 \tag{틀린 풀이} \\
&=\lim_{x\to0}\frac{\ln(1-x^2)}{x^2}=-1 \tag{맞는 풀이}
\end{align*}
로그함수 말고 지수함수로도 이러한 예가 충분히 가능하다. 언제 근사를 써도 되는지는 ``미리 보내기''에서 설명하는 내용을 참고하자.
\begin{align*}
\lim_{x\to0}\frac{e^x+e^{-x}-2}{x}&\neq\lim_{x\to0}\frac{(1+x)+(1-x)-2}{x^2}=0 \tag{틀린 풀이} \\
&=\lim_{x\to0}\frac{e^{2x}-2e^x+1}{x^2e^x}=\lim_{x\to0}\frac{1}{e^x}\left(\frac{e^x-1}{x}\right)^2=1 \tag{맞는 풀이}
\end{align*}

\subsection{곱함수의 연속 (Revisited)}
\(x=a\)에서 \(f(x)\)는 연속이고, \(g(x)\)는 제2종 불연속점을 갖는다고 하자. 일반성을 잃지 않고\footnote{다른 경우에도 비슷한 방법으로 증명 가능하다.}, 다음을 가정하자.
\[
\lim_{x\to a-}g(x)=l,\,\lim_{x\to a+}g(x)=\infty
\]
이때 두 함수의 곱 \(f(x)g(x)\)가 연속이기 위해서는 \(f(a)=0\)일 뿐만 아니라, 다음 \(0\times\infty\)꼴 극한 식이 성립해야 한다.
\[
\lim_{x\to a+}f(x)g(x)=0
\]
\(\lim_{x\to a+}g(x)=\infty\)는 \(g(x)\)의 분모에 0인수가 존재하기 때문에 생긴다고 해석하면, \(f(x)\)의 0인수 개수가 \(g(x)\)의 분모에 있는것보다 많아야 함을 알 수 있다. 예를 들어, 다음과 같이 \(g(x)\)를 정의하면
\[
g(x):=\begin{cases}
0 & (x\leq a) \\
\dfrac{b}{(x-a)^2} & (x>a)
\end{cases}
\]
\(f(x)g(x)\)가 \(x=a\)에서 연속이기 위해 \(f(x)\)가 \((x-a)\)를 인수로 2개보다 많이 가저야 한다. 만약 \(f(x)\)와 \(g(x)\)가 다항함수가 아니여서 인수를 따지기 어려운 경우, 테일러 급수를 이용한 근사를 적절히 사용하여 다항식으로 바꾸어 주면 된다.

\subsection{로피탈 정리}
함수 \(f(x)\), \(g(x)\)가 둘 다 미분가능하고, \(\lim_{x \to a}\frac{f(x)}{g(x)} = \frac{0}{0} \) 또는 \( \frac{\infty}{\infty}\) 꼴의 부정형이라면 다음이 성립한다.
\[
\lim_{x \to a}\frac{f(x)}{g(x)} = \lim_{x \to a}\frac{f'(x)}{g'(x)}
\]
특히 분자와 분모의 인수분해가 가능한 경우, 0을 만드는 요소들을 제외한 나머지에 대해 ``미리 보내기''를 수행하고 나서 로피탈 정리를 적용할 수 있다.
\[
\lim_{x\to1}\frac{(x^2+5)\ln x}{x^2+x-2}=\frac{6}{3}\times\lim_{x\to1}\frac{\ln x}{x-1}=2\times\lim_{x\to1}\frac{1}{x}=2
\]

\subsection{대칭미분가능성}
함수 \(f(x)\)에 대해 다음 극한이 존재하면, ``\(x=a\)에서 \(f(x)\)가 대칭미분 가능하다''라고 말한다.
\[
\lim_{h\to0}\frac{f(a+h)-f(a-h)}{2h}
\]
이 극한은 경우를 나누어 다음과 같이 해석할 수 있다.

\subsubsection{\texorpdfstring{\(f'(a)\)}가 존재하는 경우}
\[
\lim_{h\to0}\frac{f(a+h)-f(a-h)}{2h} = \lim_{h\to0}\left(\frac{f(a+h)-f(a)}{2h}+\frac{f(a)-f(a-h)}{2h}\right) = \frac{1}{2}(우미분계수+좌미분계수)=f'(a)
\]
따라서 \(f(x)\)가 \(x=a\)에서 미분가능하다면, 이 식은 미분계수와 같게 된다. 그러나 \(f(x)\)가 미분불가능인데 이 극한이 존재하는 경우도 있으며, 그 때 대칭미분계수의 의미도 달라지게 된다.

\subsubsection{\(x=a\)에서 \(f(x)\)가 첨점인 경우 (연속이지만 미분불가능한 경우)}
\[
\lim_{h\to0}\frac{f(a+h)-f(a-h)}{2h} = \lim_{h\to0}\frac{f(a+h)-f(a)}{2h}+\frac{f(a)-f(a-h)}{2h} =\frac{\text{(우미분계수)}+\text{(좌미분계수)}}{2}
\]

\subsubsection{\(x=a\)에서 \(f(x)\)가 제1종 불연속 (Removable)인 경우}
\(\lim_{x\to a}{f(x)}=\alpha\)라고 정의하면, 다음 식이 성립한다.
\[
\lim_{h\to0}\frac{f(a+h)-f(a-h)}{2h} = \lim_{h\to0}\frac{f(a+h)-\alpha}{2h}+\frac{f(a-h)-\alpha}{2h}
\]
따라서 이 지점에서 접선의 기울기와 유사하다.

\subsubsection{\(x=a\)에서 \(f(x)\)가 제2종 불연속 (Jump)인 경우}
\(f(x)\)가 제2종 불연속인 경우, 이 지점의 좌미분계수 또는 우미분계수 중 하나 이상은 \(\pm\infty\) 이므로 대칭미분계수가 존재하지 않는다.

\subsubsection{접선의 기울기가 \(\infty\)가 되는 경우}
이 경우 역시 대칭미분계수가 무한이 되어 존재하지 않는다.
\end{document} 
